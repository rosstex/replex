\chapter{Java API}
\label{chap:api:java}

\section{Client Library}
\label{sec:api:java:client}

HyperDex provides Java bindings in the package \code{org.hyperdex.client}.  This
package wraps the HyperDex C Client library and enables the use of native Java
data types.

This library was brought up-to-date following the 1.0.5 release.

\subsection{Building the HyperDex Java Binding}
\label{sec:api:java:building}

The HyperDex Java Binding must be requested at configure time as it is not
automatically built.  You can ensure that the Java bindings are always built by
providing the \code{--enable-java-bindings} option to \code{./configure} like
so:

\begin{consolecode}
% ./configure --enable-client --enable-java-bindings
\end{consolecode}

\subsection{Using Java Within Your Application}
\label{sec:api:java:using}

All client operation are defined in the \code{org.hyperdex.client} package.  You
can access this in your program with:

\begin{javacode}
import org.hyperdex.client.*;
\end{javacode}

\subsection{Hello World}
\label{sec:api:java:hello-world}

The following is a minimal application that stores the value "Hello World" and
then immediately retrieves the value:

\inputminted{java}{\topdir/java/client/HelloWorld.java}

You can run this example with:

\begin{consolecode}
% javac HelloWorld.java
% java -Djava.library.path=/usr/local/lib HelloWorld
put: true
got: {v=Hello World!}
\end{consolecode}

Right away, there are several points worth noting in this example:

\begin{itemize}
\item Every operation is synchronous.  The PUT and GET operations run to
completion by default.

\item Java types are automatically converted to HyperDex types.  There's no need
to specify information such as the length of each string, as one would do with
the C API.

\item We specify -Djava.library.path=/usr/local/lib.  This is necessary for
builds from source, but should not be necessary for Java bindings installed
using binary packages.
\end{itemize}

\subsection{Asynchronous Operations}
\label{sec:api:java:async-ops}

For convenience, the Java bindings treat every operation as synchronous.  This
enables you to write short programs without concern for asynchronous operations.
Most operations come with an asynchronous form, denoted by the \code{async\_}
prefix.  For example, the above Hello World example could be rewritten in
asynchronous fashion as such:

\inputminted{java}{\topdir/java/client/HelloWorldAsyncWait.java}

This enables applications to issue multiple requests simultaneously and wait for
their completion in an application-specific order.  It's also possible to use
the \code{loop} method on the client object to wait for the next request to
complete:

\inputminted{java}{\topdir/java/client/HelloWorldAsyncLoop.java}

\subsection{Data Structures}
\label{sec:api:java:data-structures}

The Java bindings automatically manage conversion of data types from Java to
HyperDex types, enabling applications to be written in idiomatic Java.

\subsubsection{Examples}
\label{sec:api:java:examples}

This section shows examples of Java data structures that are recognized by
HyperDex.  The examples here are for illustration purposes and are not
exhaustive.

\paragraph{Strings}

The HyperDex client recognizes Java's strings and automatically converts them to
HyperDex strings.  For example, the following call stores a string:
equivalent and have the same effect:

\begin{javacode}
Map<String, Object> attrs = new HashMap<String, Object>();
attrs.put("v", "someattrs");
c.put("kv", "somekey", attrs);
\end{javacode}

\paragraph{Integers}

The HyperDex client recognizes Java's integers and automatically converts them
to HyperDex integers.  For example:

\begin{javacode}
Map<String, Object> attrs = new HashMap<String, Object>();
attrs.put("v", 42);
c.put("kv", "somekey", attrs);
\end{javacode}

\paragraph{Floats}

The HyperDex client recognizes Java's floating point numbers and automatically
converts them to HyperDex floats.  For example:

\begin{javacode}
Map<String, Object> attrs = new HashMap<String, Object>();
attrs.put("v", 3.1415);
c.put("kv", "somekey", attrs);
\end{javacode}

\paragraph{Lists}

The HyperDex client recognizes Java lists and automatically converts them to
HyperDex lists.  For example:

\begin{javacode}
List<Object> list = new ArrayList<Object>();
list.add("a");
list.add("b");
list.add("c");
Map<String, Object> attrs = new HashMap<String, Object>();
attrs.put("v", list);
c.put("kv", "somekey", attrs);
\end{javacode}

\paragraph{Sets}

The HyperDex client recognizes Java sets and automatically converts them to
HyperDex sets.  For example:

\begin{javacode}
Set<Object> set = new HashSet<Object>();
set.add("a");
set.add("b");
set.add("c");
Map<String, Object> attrs = new HashMap<String, Object>();
attrs.put("v", set);
c.put("kv", "somekey", attrs);
\end{javacode}

\paragraph{Maps}

The HyperDex client recognizes Java maps and automatically converts them to
HyperDex maps.  For example:

\begin{javacode}
Map<Object, Object> map = new HashMap<Object, Object>();
map.put("k", "v");
Map<String, Object> attrs = new HashMap<String, Object>();
attrs.put("v", map);
c.put("kv", "somekey", attrs);
\end{javacode}

\subsection{Attributes}
\label{sec:api:java:attributes}

Attributes in Java are specified in the form of a map from attribute names to
their values.  As you can see in the examples above, attributes are specified in
the form:

\begin{javacode}
Map<String, Object> attrs = new HashMap<String, Object>();
\end{javacode}

\subsection{Map Attributes}
\label{sec:api:java:map-attributes}

Map attributes in Java are specified in the form of a nested map.  The outer
map key specifies the name, while the inner map key-value pair's specify the
key-value pair of the map.  For example:

\begin{javacode}
Map<String, Map<Object, Object>> mapattrs = new HashMap<String, Map<Object, Object>>();
\end{javacode}

\subsection{Predicates}
\label{sec:api:java:predicates}

Predicates in Java are specified in the form of a hash from attribute names to
their predicates.  In the simple case, the predicate is just a value to be
compared against:

\begin{javacode}
Map<String, Object> checks = new HashMap<String, Object>();
checks.put("v", "value");
\end{javacode}

This is the same as saying:

\begin{javacode}
Map<String, Object> checks = new HashMap<String, Object>();
checks.put("v", new Equals("value"));
\end{javacode}

The Java bindings support the full range of predicates supported by HyperDex
itself.  For example:

\begin{javacode}
checks.put("v", new LessEqual(5));
checks.put("v", new GreaterEqual(5));
checks.put("v", new RangeEqual(5, 10));
checks.put("v", new Regex("^s.*"));
checks.put("v", new LengthEquals(5));
checks.put("v", new LengthLessEqual(5));
checks.put("v", new LengthGreaterEqual(5));
checks.put("v", new Contains('value'));
\end{javacode}

\subsection{Error Handling}
\label{sec:api:java:error-handling}

All error handling within the Java bindings is done via the
\code{try}/\code{catch} mechanism of Java.  Errors will be thrown by the package
and should be handled by your application.  For example, if we were trying to
store an integer (5) as attribute \code{"v"}, where \code{"v"} is actually a
string, we'd generate an error.

\begin{javacode}
try
{
    attrs.put("v", 5);
    System.out.println("put: " + c.put("kv", "k", attrs));
}
catch (HyperDexClientException e)
{
    System.out.println(e.status());
    System.out.println(e.symbol());
    System.out.println(e.message());
}
\end{javacode}

Errors of type \code{HyperDexClientException} will contain both a message
indicating what went wrong, as well as the underlying \code{enum
hyperdex\_client\_returncode}.  The member \code{status} indicates the numeric
value of this enum, while \code{symbol} returns the enum as a string.  The above
code will fail with the following output:

\begin{verbatim}
8525
HYPERDEX_CLIENT_WRONGTYPE
invalid attribute "v": attribute has the wrong type
\end{verbatim}

\subsection{Operations}
\label{sec:api:java:ops}

% Copyright (c) 2014, Cornell University
% All rights reserved.
%
% Redistribution and use in source and binary forms, with or without
% modification, are permitted provided that the following conditions are met:
%
%     * Redistributions of source code must retain the above copyright notice,
%       this list of conditions and the following disclaimer.
%     * Redistributions in binary form must reproduce the above copyright
%       notice, this list of conditions and the following disclaimer in the
%       documentation and/or other materials provided with the distribution.
%     * Neither the name of HyperDex nor the names of its contributors may be
%       used to endorse or promote products derived from this software without
%       specific prior written permission.
%
% THIS SOFTWARE IS PROVIDED BY THE COPYRIGHT HOLDERS AND CONTRIBUTORS "AS IS"
% AND ANY EXPRESS OR IMPLIED WARRANTIES, INCLUDING, BUT NOT LIMITED TO, THE
% IMPLIED WARRANTIES OF MERCHANTABILITY AND FITNESS FOR A PARTICULAR PURPOSE ARE
% DISCLAIMED. IN NO EVENT SHALL THE COPYRIGHT OWNER OR CONTRIBUTORS BE LIABLE
% FOR ANY DIRECT, INDIRECT, INCIDENTAL, SPECIAL, EXEMPLARY, OR CONSEQUENTIAL
% DAMAGES (INCLUDING, BUT NOT LIMITED TO, PROCUREMENT OF SUBSTITUTE GOODS OR
% SERVICES; LOSS OF USE, DATA, OR PROFITS; OR BUSINESS INTERRUPTION) HOWEVER
% CAUSED AND ON ANY THEORY OF LIABILITY, WHETHER IN CONTRACT, STRICT LIABILITY,
% OR TORT (INCLUDING NEGLIGENCE OR OTHERWISE) ARISING IN ANY WAY OUT OF THE USE
% OF THIS SOFTWARE, EVEN IF ADVISED OF THE POSSIBILITY OF SUCH DAMAGE.

% This LaTeX file is generated by bindings/go.py

%%%%%%%%%%%%%%%%%%%% Get %%%%%%%%%%%%%%%%%%%%
\pagebreak
\subsubsection{\code{Get}}
\label{api:Go:Get}
\index{Get!Go API}
Retrieve an object from \code{space} using \code{key}.


\paragraph{Definition:}
\begin{gocode}
func (client *Client) Get(spacename string, key Value) (attrs Attributes, err *Error)
\end{gocode}

\paragraph{Parameters:}
\begin{itemize}[noitemsep]
\item \code{spacename}\\
The name of the space as a C-string.

\item \code{key}\\
The key for the operation where \code{key} is a Javascript value.

\end{itemize}

\paragraph{Returns:}
This function returns an object indicating the success or failure of the
operation.  Valid values to be returned are:

\begin{itemize}[noitemsep]
\item The returned object, represented as a map from attributes' names to Java
    objects.
\item \code{False} if any provided predicates failed.
\item \code{null} if the operation requires an existing value and none exists
\end{itemize}

On error, this function will raise a \code{HyperDexClientException} describing
the error.


%%%%%%%%%%%%%%%%%%%% GetPartial %%%%%%%%%%%%%%%%%%%%
\pagebreak
\subsubsection{\code{GetPartial}}
\label{api:Go:GetPartial}
\index{GetPartial!Go API}
Retrieve part of an object from \code{space} using \code{key}.  Only the
specified attributes of the object will be returned to the client.


\paragraph{Definition:}
\begin{gocode}
func (client *Client) GetPartial(spacename string, key Value, attributenames AttributeNames) (attrs Attributes, err *Error)
\end{gocode}

\paragraph{Parameters:}
\begin{itemize}[noitemsep]
\item \code{spacename}\\
The name of the space as a C-string.

\item \code{key}\\
The key for the operation where \code{key} is a Javascript value.

\item \code{attributenames}\\
A list of attributes to return.  \code{attrnames} is a \code{List<String>}.

\end{itemize}

\paragraph{Returns:}
This function returns an object indicating the success or failure of the
operation.  Valid values to be returned are:

\begin{itemize}[noitemsep]
\item The returned object, represented as a map from attributes' names to Java
    objects.
\item \code{False} if any provided predicates failed.
\item \code{null} if the operation requires an existing value and none exists
\end{itemize}

On error, this function will raise a \code{HyperDexClientException} describing
the error.


%%%%%%%%%%%%%%%%%%%% Put %%%%%%%%%%%%%%%%%%%%
\pagebreak
\subsubsection{\code{Put}}
\label{api:Go:Put}
\index{Put!Go API}
Store or update an object by key.  The object's attributes will be set to the
values specified by \code{attrs}.
If the object exists, it will be updated and all existing values not altered by
\code{attrs} will be preserved.  If the object does not exist, a new object will
be created, with its attributes initialized to their default values.



\paragraph{Definition:}
\begin{gocode}
func (client *Client) Put(spacename string, key Value, attributes Attributes) (err *Error)
\end{gocode}

\paragraph{Parameters:}
\begin{itemize}[noitemsep]
\item \code{spacename}\\
The name of the space as a C-string.

\item \code{key}\\
The key for the operation where \code{key} is a Javascript value.

\item \code{attributes}\\
The set of attributes to modify and their respective values.  \code{attrs}
points to an array of length \code{attrs\_sz}.

\end{itemize}

\paragraph{Returns:}
This function returns via the provided callback.  In the normal case, the first
argument will indicate success or failure of the operation with one of the
following values:

\begin{itemize}[noitemsep]
\item \code{true} if the operation succeeded
\item \code{false} if any provided predicates failed
\item \code{null} if the operation requires an existing value and none exist
\end{itemize}

If the operation encounters any error, the error argument will be provided and
will specify the error, in which case the first argument is undefined.


%%%%%%%%%%%%%%%%%%%% CondPut %%%%%%%%%%%%%%%%%%%%
\pagebreak
\subsubsection{\code{CondPut}}
\label{api:Go:CondPut}
\index{CondPut!Go API}
Conditionally update an the object stored under \code{key} in \code{space}.
Existing values will be overwitten with the values specified by \code{attrs}.
Values not specified by \code{attrs} will remain unchanged.
This operation requires a pre-existing object in order to complete successfully.
If no object exists, the operation will fail with \code{NOTFOUND}.


This operation will succeed if and only if the predicates specified by
\code{checks} hold on the pre-existing object.  If any of the predicates are not
true for the existing object, then the operation will have no effect and fail
with \code{CMPFAIL}.

All checks are atomic with the write.  HyperDex guarantees that no other
operation will come between validating the checks, and writing the new version
of the object..



\paragraph{Definition:}
\begin{gocode}
func (client *Client) CondPut(spacename string, key Value, predicates []Predicate, attributes Attributes) (err *Error)
\end{gocode}

\paragraph{Parameters:}
\begin{itemize}[noitemsep]
\item \code{spacename}\\
The name of the space as a C-string.

\item \code{key}\\
The key for the operation where \code{key} is a Javascript value.

\item \code{predicates}\\
A set of predicates to check against.  \code{checks} is a map from the
attributes' names to the predicates to check.

\item \code{attributes}\\
The set of attributes to modify and their respective values.  \code{attrs}
points to an array of length \code{attrs\_sz}.

\end{itemize}

\paragraph{Returns:}
This function returns via the provided callback.  In the normal case, the first
argument will indicate success or failure of the operation with one of the
following values:

\begin{itemize}[noitemsep]
\item \code{true} if the operation succeeded
\item \code{false} if any provided predicates failed
\item \code{null} if the operation requires an existing value and none exist
\end{itemize}

If the operation encounters any error, the error argument will be provided and
will specify the error, in which case the first argument is undefined.


%%%%%%%%%%%%%%%%%%%% CondPutOrCreate %%%%%%%%%%%%%%%%%%%%
\pagebreak
\subsubsection{\code{CondPutOrCreate}}
\label{api:Go:CondPutOrCreate}
\index{CondPutOrCreate!Go API}
Conditionally update an the object stored under \code{key} in \code{space}.
Existing values will be overwitten with the values specified by \code{attrs}.
Values not specified by \code{attrs} will remain unchanged.  If the object
exists, this is equivalent to a \code{cond\_put}.  If the object does not exist,
this is equivalent to a \code{put}

This operation will succeed if and only if the predicates specified by
\code{checks} hold on the pre-existing object.  If any of the predicates are not
true for the existing object, then the operation will have no effect and fail
with \code{CMPFAIL}.

All checks are atomic with the write.  HyperDex guarantees that no other
operation will come between validating the checks, and writing the new version
of the object..



\paragraph{Definition:}
\begin{gocode}
func (client *Client) CondPutOrCreate(spacename string, key Value, predicates []Predicate, attributes Attributes) (err *Error)
\end{gocode}

\paragraph{Parameters:}
\begin{itemize}[noitemsep]
\item \code{spacename}\\
The name of the space as a C-string.

\item \code{key}\\
The key for the operation where \code{key} is a Javascript value.

\item \code{predicates}\\
A set of predicates to check against.  \code{checks} is a map from the
attributes' names to the predicates to check.

\item \code{attributes}\\
The set of attributes to modify and their respective values.  \code{attrs}
points to an array of length \code{attrs\_sz}.

\end{itemize}

\paragraph{Returns:}
This function returns via the provided callback.  In the normal case, the first
argument will indicate success or failure of the operation with one of the
following values:

\begin{itemize}[noitemsep]
\item \code{true} if the operation succeeded
\item \code{false} if any provided predicates failed
\item \code{null} if the operation requires an existing value and none exist
\end{itemize}

If the operation encounters any error, the error argument will be provided and
will specify the error, in which case the first argument is undefined.


%%%%%%%%%%%%%%%%%%%% GroupPut %%%%%%%%%%%%%%%%%%%%
\pagebreak
\subsubsection{\code{GroupPut}}
\label{api:Go:GroupPut}
\index{GroupPut!Go API}
Update all objects stored in \code{space} that match \code{checks}.  Existing
values will be overwitten with the values specified by \code{attrs}.  Values not
specified by \code{attrs} will remain unchanged.

This operation will only affect objects that match the provided \code{checks}.
Objects that do not match \code{checks} will be unaffected by the group call.
Each object that matches \code{checks} will be atomically updated with the check
on the object.  HyperDex guarantees that no object will be altered if the
\code{checks} do not pass at the time of the write.  Objects that are updated
concurrently with the group call may or may not be updated; however, regardless
of any other concurrent operations, the preceding guarantee will always hold.



\paragraph{Definition:}
\begin{gocode}
func (client *Client) GroupPut(spacename string, predicates []Predicate, attributes Attributes) (count uint64, err *Error)
\end{gocode}

\paragraph{Parameters:}
\begin{itemize}[noitemsep]
\item \code{spacename}\\
The name of the space as a C-string.

\item \code{predicates}\\
A set of predicates to check against.  \code{checks} is a map from the
attributes' names to the predicates to check.

\item \code{attributes}\\
The set of attributes to modify and their respective values.  \code{attrs}
points to an array of length \code{attrs\_sz}.

\end{itemize}

\paragraph{Returns:}
This function returns via the provided callback.  In the normal case, the first
argument will be the total number of objects that match the predicate.

If the operation encounters any error, the error argument will be provided and
will specify the error, in which case the first argument is undefined.


%%%%%%%%%%%%%%%%%%%% PutIfNotExist %%%%%%%%%%%%%%%%%%%%
\pagebreak
\subsubsection{\code{PutIfNotExist}}
\label{api:Go:PutIfNotExist}
\index{PutIfNotExist!Go API}
Store or object under \code{key} in \code{space} if and only if the operation
creates a new object.  The object's attributes will be set to the values
specified by \code{attrs}; any attributes not specified by \code{attrs} will be
initialized to their defaults.  If the object exists, the operation will fail
with \code{CMPFAIL}.


\paragraph{Definition:}
\begin{gocode}
func (client *Client) PutIfNotExist(spacename string, key Value, attributes Attributes) (err *Error)
\end{gocode}

\paragraph{Parameters:}
\begin{itemize}[noitemsep]
\item \code{spacename}\\
The name of the space as a C-string.

\item \code{key}\\
The key for the operation where \code{key} is a Javascript value.

\item \code{attributes}\\
The set of attributes to modify and their respective values.  \code{attrs}
points to an array of length \code{attrs\_sz}.

\end{itemize}

\paragraph{Returns:}
This function returns via the provided callback.  In the normal case, the first
argument will indicate success or failure of the operation with one of the
following values:

\begin{itemize}[noitemsep]
\item \code{true} if the operation succeeded
\item \code{false} if any provided predicates failed
\item \code{null} if the operation requires an existing value and none exist
\end{itemize}

If the operation encounters any error, the error argument will be provided and
will specify the error, in which case the first argument is undefined.


%%%%%%%%%%%%%%%%%%%% Del %%%%%%%%%%%%%%%%%%%%
\pagebreak
\subsubsection{\code{Del}}
\label{api:Go:Del}
\index{Del!Go API}
Delete \code{key} from \code{space}.
If no object exists, the operation will fail with \code{NOTFOUND}.



\paragraph{Definition:}
\begin{gocode}
func (client *Client) Del(spacename string, key Value) (err *Error)
\end{gocode}

\paragraph{Parameters:}
\begin{itemize}[noitemsep]
\item \code{spacename}\\
The name of the space as a C-string.

\item \code{key}\\
The key for the operation where \code{key} is a Javascript value.

\end{itemize}

\paragraph{Returns:}
This function returns via the provided callback.  In the normal case, the first
argument will indicate success or failure of the operation with one of the
following values:

\begin{itemize}[noitemsep]
\item \code{true} if the operation succeeded
\item \code{false} if any provided predicates failed
\item \code{null} if the operation requires an existing value and none exist
\end{itemize}

If the operation encounters any error, the error argument will be provided and
will specify the error, in which case the first argument is undefined.


%%%%%%%%%%%%%%%%%%%% CondDel %%%%%%%%%%%%%%%%%%%%
\pagebreak
\subsubsection{\code{CondDel}}
\label{api:Go:CondDel}
\index{CondDel!Go API}
Conditionally delete the object stored under \code{key} from \code{space}.
If no object exists, the operation will fail with \code{NOTFOUND}.


This operation will succeed if and only if the predicates specified by
\code{checks} hold on the pre-existing object.  If any of the predicates are not
true for the existing object, then the operation will have no effect and fail
with \code{CMPFAIL}.

All checks are atomic with the write.  HyperDex guarantees that no other
operation will come between validating the checks, and writing the new version
of the object..



\paragraph{Definition:}
\begin{gocode}
func (client *Client) CondDel(spacename string, key Value, predicates []Predicate) (err *Error)
\end{gocode}

\paragraph{Parameters:}
\begin{itemize}[noitemsep]
\item \code{spacename}\\
The name of the space as a C-string.

\item \code{key}\\
The key for the operation where \code{key} is a Javascript value.

\item \code{predicates}\\
A set of predicates to check against.  \code{checks} is a map from the
attributes' names to the predicates to check.

\end{itemize}

\paragraph{Returns:}
This function returns via the provided callback.  In the normal case, the first
argument will indicate success or failure of the operation with one of the
following values:

\begin{itemize}[noitemsep]
\item \code{true} if the operation succeeded
\item \code{false} if any provided predicates failed
\item \code{null} if the operation requires an existing value and none exist
\end{itemize}

If the operation encounters any error, the error argument will be provided and
will specify the error, in which case the first argument is undefined.


%%%%%%%%%%%%%%%%%%%% GroupDel %%%%%%%%%%%%%%%%%%%%
\pagebreak
\subsubsection{\code{GroupDel}}
\label{api:Go:GroupDel}
\index{GroupDel!Go API}
Delete all objects that match the specified \code{checks}.

This operation will only affect objects that match the provided \code{checks}.
Objects that do not match \code{checks} will be unaffected by the group call.
Each object that matches \code{checks} will be atomically updated with the check
on the object.  HyperDex guarantees that no object will be altered if the
\code{checks} do not pass at the time of the write.  Objects that are updated
concurrently with the group call may or may not be updated; however, regardless
of any other concurrent operations, the preceding guarantee will always hold.



\paragraph{Definition:}
\begin{gocode}
func (client *Client) GroupDel(spacename string, predicates []Predicate) (count uint64, err *Error)
\end{gocode}

\paragraph{Parameters:}
\begin{itemize}[noitemsep]
\item \code{spacename}\\
The name of the space as a C-string.

\item \code{predicates}\\
A set of predicates to check against.  \code{checks} is a map from the
attributes' names to the predicates to check.

\end{itemize}

\paragraph{Returns:}
This function returns via the provided callback.  In the normal case, the first
argument will be the total number of objects that match the predicate.

If the operation encounters any error, the error argument will be provided and
will specify the error, in which case the first argument is undefined.


%%%%%%%%%%%%%%%%%%%% AtomicAdd %%%%%%%%%%%%%%%%%%%%
\pagebreak
\subsubsection{\code{AtomicAdd}}
\label{api:Go:AtomicAdd}
\index{AtomicAdd!Go API}
Add the specified number to the existing value for each attribute.
This operation requires a pre-existing object in order to complete successfully.
If no object exists, the operation will fail with \code{NOTFOUND}.



\paragraph{Definition:}
\begin{gocode}
func (client *Client) AtomicAdd(spacename string, key Value, attributes Attributes) (err *Error)
\end{gocode}

\paragraph{Parameters:}
\begin{itemize}[noitemsep]
\item \code{spacename}\\
The name of the space as a C-string.

\item \code{key}\\
The key for the operation where \code{key} is a Javascript value.

\item \code{attributes}\\
The set of attributes to modify and their respective values.  \code{attrs}
points to an array of length \code{attrs\_sz}.

\end{itemize}

\paragraph{Returns:}
This function returns via the provided callback.  In the normal case, the first
argument will indicate success or failure of the operation with one of the
following values:

\begin{itemize}[noitemsep]
\item \code{true} if the operation succeeded
\item \code{false} if any provided predicates failed
\item \code{null} if the operation requires an existing value and none exist
\end{itemize}

If the operation encounters any error, the error argument will be provided and
will specify the error, in which case the first argument is undefined.


%%%%%%%%%%%%%%%%%%%% CondAtomicAdd %%%%%%%%%%%%%%%%%%%%
\pagebreak
\subsubsection{\code{CondAtomicAdd}}
\label{api:Go:CondAtomicAdd}
\index{CondAtomicAdd!Go API}
Add the specified number to the existing value for each attribute if and only if
the \code{checks} hold on the object.
This operation requires a pre-existing object in order to complete successfully.
If no object exists, the operation will fail with \code{NOTFOUND}.


This operation will succeed if and only if the predicates specified by
\code{checks} hold on the pre-existing object.  If any of the predicates are not
true for the existing object, then the operation will have no effect and fail
with \code{CMPFAIL}.

All checks are atomic with the write.  HyperDex guarantees that no other
operation will come between validating the checks, and writing the new version
of the object..



\paragraph{Definition:}
\begin{gocode}
func (client *Client) CondAtomicAdd(spacename string, key Value, predicates []Predicate, attributes Attributes) (err *Error)
\end{gocode}

\paragraph{Parameters:}
\begin{itemize}[noitemsep]
\item \code{spacename}\\
The name of the space as a C-string.

\item \code{key}\\
The key for the operation where \code{key} is a Javascript value.

\item \code{predicates}\\
A set of predicates to check against.  \code{checks} is a map from the
attributes' names to the predicates to check.

\item \code{attributes}\\
The set of attributes to modify and their respective values.  \code{attrs}
points to an array of length \code{attrs\_sz}.

\end{itemize}

\paragraph{Returns:}
This function returns via the provided callback.  In the normal case, the first
argument will indicate success or failure of the operation with one of the
following values:

\begin{itemize}[noitemsep]
\item \code{true} if the operation succeeded
\item \code{false} if any provided predicates failed
\item \code{null} if the operation requires an existing value and none exist
\end{itemize}

If the operation encounters any error, the error argument will be provided and
will specify the error, in which case the first argument is undefined.


%%%%%%%%%%%%%%%%%%%% GroupAtomicAdd %%%%%%%%%%%%%%%%%%%%
\pagebreak
\subsubsection{\code{GroupAtomicAdd}}
\label{api:Go:GroupAtomicAdd}
\index{GroupAtomicAdd!Go API}
Add the specified number to the existing value for each object in \code{space}
that matches \code{checks}.

This operation will only affect objects that match the provided \code{checks}.
Objects that do not match \code{checks} will be unaffected by the group call.
Each object that matches \code{checks} will be atomically updated with the check
on the object.  HyperDex guarantees that no object will be altered if the
\code{checks} do not pass at the time of the write.  Objects that are updated
concurrently with the group call may or may not be updated; however, regardless
of any other concurrent operations, the preceding guarantee will always hold.



\paragraph{Definition:}
\begin{gocode}
func (client *Client) GroupAtomicAdd(spacename string, predicates []Predicate, attributes Attributes) (count uint64, err *Error)
\end{gocode}

\paragraph{Parameters:}
\begin{itemize}[noitemsep]
\item \code{spacename}\\
The name of the space as a C-string.

\item \code{predicates}\\
A set of predicates to check against.  \code{checks} is a map from the
attributes' names to the predicates to check.

\item \code{attributes}\\
The set of attributes to modify and their respective values.  \code{attrs}
points to an array of length \code{attrs\_sz}.

\end{itemize}

\paragraph{Returns:}
This function returns via the provided callback.  In the normal case, the first
argument will be the total number of objects that match the predicate.

If the operation encounters any error, the error argument will be provided and
will specify the error, in which case the first argument is undefined.


%%%%%%%%%%%%%%%%%%%% AtomicSub %%%%%%%%%%%%%%%%%%%%
\pagebreak
\subsubsection{\code{AtomicSub}}
\label{api:Go:AtomicSub}
\index{AtomicSub!Go API}
Subtract the specified number from the existing value for each attribute.
This operation requires a pre-existing object in order to complete successfully.
If no object exists, the operation will fail with \code{NOTFOUND}.



\paragraph{Definition:}
\begin{gocode}
func (client *Client) AtomicSub(spacename string, key Value, attributes Attributes) (err *Error)
\end{gocode}

\paragraph{Parameters:}
\begin{itemize}[noitemsep]
\item \code{spacename}\\
The name of the space as a C-string.

\item \code{key}\\
The key for the operation where \code{key} is a Javascript value.

\item \code{attributes}\\
The set of attributes to modify and their respective values.  \code{attrs}
points to an array of length \code{attrs\_sz}.

\end{itemize}

\paragraph{Returns:}
This function returns via the provided callback.  In the normal case, the first
argument will indicate success or failure of the operation with one of the
following values:

\begin{itemize}[noitemsep]
\item \code{true} if the operation succeeded
\item \code{false} if any provided predicates failed
\item \code{null} if the operation requires an existing value and none exist
\end{itemize}

If the operation encounters any error, the error argument will be provided and
will specify the error, in which case the first argument is undefined.


%%%%%%%%%%%%%%%%%%%% CondAtomicSub %%%%%%%%%%%%%%%%%%%%
\pagebreak
\subsubsection{\code{CondAtomicSub}}
\label{api:Go:CondAtomicSub}
\index{CondAtomicSub!Go API}
Subtract the specified number from the existing value for each attribute if and
only if the \code{checks} hold on the object.
This operation requires a pre-existing object in order to complete successfully.
If no object exists, the operation will fail with \code{NOTFOUND}.


This operation will succeed if and only if the predicates specified by
\code{checks} hold on the pre-existing object.  If any of the predicates are not
true for the existing object, then the operation will have no effect and fail
with \code{CMPFAIL}.

All checks are atomic with the write.  HyperDex guarantees that no other
operation will come between validating the checks, and writing the new version
of the object..



\paragraph{Definition:}
\begin{gocode}
func (client *Client) CondAtomicSub(spacename string, key Value, predicates []Predicate, attributes Attributes) (err *Error)
\end{gocode}

\paragraph{Parameters:}
\begin{itemize}[noitemsep]
\item \code{spacename}\\
The name of the space as a C-string.

\item \code{key}\\
The key for the operation where \code{key} is a Javascript value.

\item \code{predicates}\\
A set of predicates to check against.  \code{checks} is a map from the
attributes' names to the predicates to check.

\item \code{attributes}\\
The set of attributes to modify and their respective values.  \code{attrs}
points to an array of length \code{attrs\_sz}.

\end{itemize}

\paragraph{Returns:}
This function returns via the provided callback.  In the normal case, the first
argument will indicate success or failure of the operation with one of the
following values:

\begin{itemize}[noitemsep]
\item \code{true} if the operation succeeded
\item \code{false} if any provided predicates failed
\item \code{null} if the operation requires an existing value and none exist
\end{itemize}

If the operation encounters any error, the error argument will be provided and
will specify the error, in which case the first argument is undefined.


%%%%%%%%%%%%%%%%%%%% GroupAtomicSub %%%%%%%%%%%%%%%%%%%%
\pagebreak
\subsubsection{\code{GroupAtomicSub}}
\label{api:Go:GroupAtomicSub}
\index{GroupAtomicSub!Go API}
Subtract the specified number from the existing value for each object in
\code{space} that matches \code{checks}.

This operation will only affect objects that match the provided \code{checks}.
Objects that do not match \code{checks} will be unaffected by the group call.
Each object that matches \code{checks} will be atomically updated with the check
on the object.  HyperDex guarantees that no object will be altered if the
\code{checks} do not pass at the time of the write.  Objects that are updated
concurrently with the group call may or may not be updated; however, regardless
of any other concurrent operations, the preceding guarantee will always hold.



\paragraph{Definition:}
\begin{gocode}
func (client *Client) GroupAtomicSub(spacename string, predicates []Predicate, attributes Attributes) (count uint64, err *Error)
\end{gocode}

\paragraph{Parameters:}
\begin{itemize}[noitemsep]
\item \code{spacename}\\
The name of the space as a C-string.

\item \code{predicates}\\
A set of predicates to check against.  \code{checks} is a map from the
attributes' names to the predicates to check.

\item \code{attributes}\\
The set of attributes to modify and their respective values.  \code{attrs}
points to an array of length \code{attrs\_sz}.

\end{itemize}

\paragraph{Returns:}
This function returns via the provided callback.  In the normal case, the first
argument will be the total number of objects that match the predicate.

If the operation encounters any error, the error argument will be provided and
will specify the error, in which case the first argument is undefined.


%%%%%%%%%%%%%%%%%%%% AtomicMul %%%%%%%%%%%%%%%%%%%%
\pagebreak
\subsubsection{\code{AtomicMul}}
\label{api:Go:AtomicMul}
\index{AtomicMul!Go API}
Multiply the existing value by the specified number for each attribute.
This operation requires a pre-existing object in order to complete successfully.
If no object exists, the operation will fail with \code{NOTFOUND}.



\paragraph{Definition:}
\begin{gocode}
func (client *Client) AtomicMul(spacename string, key Value, attributes Attributes) (err *Error)
\end{gocode}

\paragraph{Parameters:}
\begin{itemize}[noitemsep]
\item \code{spacename}\\
The name of the space as a C-string.

\item \code{key}\\
The key for the operation where \code{key} is a Javascript value.

\item \code{attributes}\\
The set of attributes to modify and their respective values.  \code{attrs}
points to an array of length \code{attrs\_sz}.

\end{itemize}

\paragraph{Returns:}
This function returns via the provided callback.  In the normal case, the first
argument will indicate success or failure of the operation with one of the
following values:

\begin{itemize}[noitemsep]
\item \code{true} if the operation succeeded
\item \code{false} if any provided predicates failed
\item \code{null} if the operation requires an existing value and none exist
\end{itemize}

If the operation encounters any error, the error argument will be provided and
will specify the error, in which case the first argument is undefined.


%%%%%%%%%%%%%%%%%%%% CondAtomicMul %%%%%%%%%%%%%%%%%%%%
\pagebreak
\subsubsection{\code{CondAtomicMul}}
\label{api:Go:CondAtomicMul}
\index{CondAtomicMul!Go API}
Multiply the existing value by the specified number for each attribute if and
only if the \code{checks} hold on the object.
This operation requires a pre-existing object in order to complete successfully.
If no object exists, the operation will fail with \code{NOTFOUND}.


This operation will succeed if and only if the predicates specified by
\code{checks} hold on the pre-existing object.  If any of the predicates are not
true for the existing object, then the operation will have no effect and fail
with \code{CMPFAIL}.

All checks are atomic with the write.  HyperDex guarantees that no other
operation will come between validating the checks, and writing the new version
of the object..



\paragraph{Definition:}
\begin{gocode}
func (client *Client) CondAtomicMul(spacename string, key Value, predicates []Predicate, attributes Attributes) (err *Error)
\end{gocode}

\paragraph{Parameters:}
\begin{itemize}[noitemsep]
\item \code{spacename}\\
The name of the space as a C-string.

\item \code{key}\\
The key for the operation where \code{key} is a Javascript value.

\item \code{predicates}\\
A set of predicates to check against.  \code{checks} is a map from the
attributes' names to the predicates to check.

\item \code{attributes}\\
The set of attributes to modify and their respective values.  \code{attrs}
points to an array of length \code{attrs\_sz}.

\end{itemize}

\paragraph{Returns:}
This function returns via the provided callback.  In the normal case, the first
argument will indicate success or failure of the operation with one of the
following values:

\begin{itemize}[noitemsep]
\item \code{true} if the operation succeeded
\item \code{false} if any provided predicates failed
\item \code{null} if the operation requires an existing value and none exist
\end{itemize}

If the operation encounters any error, the error argument will be provided and
will specify the error, in which case the first argument is undefined.


%%%%%%%%%%%%%%%%%%%% GroupAtomicMul %%%%%%%%%%%%%%%%%%%%
\pagebreak
\subsubsection{\code{GroupAtomicMul}}
\label{api:Go:GroupAtomicMul}
\index{GroupAtomicMul!Go API}
Multiply the existing value by the specified number for each object in
\code{space} that matches \code{checks}.

This operation will only affect objects that match the provided \code{checks}.
Objects that do not match \code{checks} will be unaffected by the group call.
Each object that matches \code{checks} will be atomically updated with the check
on the object.  HyperDex guarantees that no object will be altered if the
\code{checks} do not pass at the time of the write.  Objects that are updated
concurrently with the group call may or may not be updated; however, regardless
of any other concurrent operations, the preceding guarantee will always hold.



\paragraph{Definition:}
\begin{gocode}
func (client *Client) GroupAtomicMul(spacename string, predicates []Predicate, attributes Attributes) (count uint64, err *Error)
\end{gocode}

\paragraph{Parameters:}
\begin{itemize}[noitemsep]
\item \code{spacename}\\
The name of the space as a C-string.

\item \code{predicates}\\
A set of predicates to check against.  \code{checks} is a map from the
attributes' names to the predicates to check.

\item \code{attributes}\\
The set of attributes to modify and their respective values.  \code{attrs}
points to an array of length \code{attrs\_sz}.

\end{itemize}

\paragraph{Returns:}
This function returns via the provided callback.  In the normal case, the first
argument will be the total number of objects that match the predicate.

If the operation encounters any error, the error argument will be provided and
will specify the error, in which case the first argument is undefined.


%%%%%%%%%%%%%%%%%%%% AtomicDiv %%%%%%%%%%%%%%%%%%%%
\pagebreak
\subsubsection{\code{AtomicDiv}}
\label{api:Go:AtomicDiv}
\index{AtomicDiv!Go API}
Divide the existing value by the specified number for each attribute.
This operation requires a pre-existing object in order to complete successfully.
If no object exists, the operation will fail with \code{NOTFOUND}.



\paragraph{Definition:}
\begin{gocode}
func (client *Client) AtomicDiv(spacename string, key Value, attributes Attributes) (err *Error)
\end{gocode}

\paragraph{Parameters:}
\begin{itemize}[noitemsep]
\item \code{spacename}\\
The name of the space as a C-string.

\item \code{key}\\
The key for the operation where \code{key} is a Javascript value.

\item \code{attributes}\\
The set of attributes to modify and their respective values.  \code{attrs}
points to an array of length \code{attrs\_sz}.

\end{itemize}

\paragraph{Returns:}
This function returns via the provided callback.  In the normal case, the first
argument will indicate success or failure of the operation with one of the
following values:

\begin{itemize}[noitemsep]
\item \code{true} if the operation succeeded
\item \code{false} if any provided predicates failed
\item \code{null} if the operation requires an existing value and none exist
\end{itemize}

If the operation encounters any error, the error argument will be provided and
will specify the error, in which case the first argument is undefined.


%%%%%%%%%%%%%%%%%%%% CondAtomicDiv %%%%%%%%%%%%%%%%%%%%
\pagebreak
\subsubsection{\code{CondAtomicDiv}}
\label{api:Go:CondAtomicDiv}
\index{CondAtomicDiv!Go API}
Divide the existing value by the specified number for each attribute if and only
if the \code{checks} hold on the object.
This operation requires a pre-existing object in order to complete successfully.
If no object exists, the operation will fail with \code{NOTFOUND}.


This operation will succeed if and only if the predicates specified by
\code{checks} hold on the pre-existing object.  If any of the predicates are not
true for the existing object, then the operation will have no effect and fail
with \code{CMPFAIL}.

All checks are atomic with the write.  HyperDex guarantees that no other
operation will come between validating the checks, and writing the new version
of the object..



\paragraph{Definition:}
\begin{gocode}
func (client *Client) CondAtomicDiv(spacename string, key Value, predicates []Predicate, attributes Attributes) (err *Error)
\end{gocode}

\paragraph{Parameters:}
\begin{itemize}[noitemsep]
\item \code{spacename}\\
The name of the space as a C-string.

\item \code{key}\\
The key for the operation where \code{key} is a Javascript value.

\item \code{predicates}\\
A set of predicates to check against.  \code{checks} is a map from the
attributes' names to the predicates to check.

\item \code{attributes}\\
The set of attributes to modify and their respective values.  \code{attrs}
points to an array of length \code{attrs\_sz}.

\end{itemize}

\paragraph{Returns:}
This function returns via the provided callback.  In the normal case, the first
argument will indicate success or failure of the operation with one of the
following values:

\begin{itemize}[noitemsep]
\item \code{true} if the operation succeeded
\item \code{false} if any provided predicates failed
\item \code{null} if the operation requires an existing value and none exist
\end{itemize}

If the operation encounters any error, the error argument will be provided and
will specify the error, in which case the first argument is undefined.


%%%%%%%%%%%%%%%%%%%% GroupAtomicDiv %%%%%%%%%%%%%%%%%%%%
\pagebreak
\subsubsection{\code{GroupAtomicDiv}}
\label{api:Go:GroupAtomicDiv}
\index{GroupAtomicDiv!Go API}
Divide the existing value by the specified number for each object in
\code{space} that matches \code{checks}.

This operation will only affect objects that match the provided \code{checks}.
Objects that do not match \code{checks} will be unaffected by the group call.
Each object that matches \code{checks} will be atomically updated with the check
on the object.  HyperDex guarantees that no object will be altered if the
\code{checks} do not pass at the time of the write.  Objects that are updated
concurrently with the group call may or may not be updated; however, regardless
of any other concurrent operations, the preceding guarantee will always hold.



\paragraph{Definition:}
\begin{gocode}
func (client *Client) GroupAtomicDiv(spacename string, predicates []Predicate, attributes Attributes) (count uint64, err *Error)
\end{gocode}

\paragraph{Parameters:}
\begin{itemize}[noitemsep]
\item \code{spacename}\\
The name of the space as a C-string.

\item \code{predicates}\\
A set of predicates to check against.  \code{checks} is a map from the
attributes' names to the predicates to check.

\item \code{attributes}\\
The set of attributes to modify and their respective values.  \code{attrs}
points to an array of length \code{attrs\_sz}.

\end{itemize}

\paragraph{Returns:}
This function returns via the provided callback.  In the normal case, the first
argument will be the total number of objects that match the predicate.

If the operation encounters any error, the error argument will be provided and
will specify the error, in which case the first argument is undefined.


%%%%%%%%%%%%%%%%%%%% AtomicMod %%%%%%%%%%%%%%%%%%%%
\pagebreak
\subsubsection{\code{AtomicMod}}
\label{api:Go:AtomicMod}
\index{AtomicMod!Go API}
Store the existing value modulo the specified number for each attribute.
This operation requires a pre-existing object in order to complete successfully.
If no object exists, the operation will fail with \code{NOTFOUND}.



\paragraph{Definition:}
\begin{gocode}
func (client *Client) AtomicMod(spacename string, key Value, attributes Attributes) (err *Error)
\end{gocode}

\paragraph{Parameters:}
\begin{itemize}[noitemsep]
\item \code{spacename}\\
The name of the space as a C-string.

\item \code{key}\\
The key for the operation where \code{key} is a Javascript value.

\item \code{attributes}\\
The set of attributes to modify and their respective values.  \code{attrs}
points to an array of length \code{attrs\_sz}.

\end{itemize}

\paragraph{Returns:}
This function returns via the provided callback.  In the normal case, the first
argument will indicate success or failure of the operation with one of the
following values:

\begin{itemize}[noitemsep]
\item \code{true} if the operation succeeded
\item \code{false} if any provided predicates failed
\item \code{null} if the operation requires an existing value and none exist
\end{itemize}

If the operation encounters any error, the error argument will be provided and
will specify the error, in which case the first argument is undefined.


%%%%%%%%%%%%%%%%%%%% CondAtomicMod %%%%%%%%%%%%%%%%%%%%
\pagebreak
\subsubsection{\code{CondAtomicMod}}
\label{api:Go:CondAtomicMod}
\index{CondAtomicMod!Go API}
Store the existing value modulo the specified number for each attribute if and
only if the \code{checks} hold on the object.
This operation requires a pre-existing object in order to complete successfully.
If no object exists, the operation will fail with \code{NOTFOUND}.


This operation will succeed if and only if the predicates specified by
\code{checks} hold on the pre-existing object.  If any of the predicates are not
true for the existing object, then the operation will have no effect and fail
with \code{CMPFAIL}.

All checks are atomic with the write.  HyperDex guarantees that no other
operation will come between validating the checks, and writing the new version
of the object..



\paragraph{Definition:}
\begin{gocode}
func (client *Client) CondAtomicMod(spacename string, key Value, predicates []Predicate, attributes Attributes) (err *Error)
\end{gocode}

\paragraph{Parameters:}
\begin{itemize}[noitemsep]
\item \code{spacename}\\
The name of the space as a C-string.

\item \code{key}\\
The key for the operation where \code{key} is a Javascript value.

\item \code{predicates}\\
A set of predicates to check against.  \code{checks} is a map from the
attributes' names to the predicates to check.

\item \code{attributes}\\
The set of attributes to modify and their respective values.  \code{attrs}
points to an array of length \code{attrs\_sz}.

\end{itemize}

\paragraph{Returns:}
This function returns via the provided callback.  In the normal case, the first
argument will indicate success or failure of the operation with one of the
following values:

\begin{itemize}[noitemsep]
\item \code{true} if the operation succeeded
\item \code{false} if any provided predicates failed
\item \code{null} if the operation requires an existing value and none exist
\end{itemize}

If the operation encounters any error, the error argument will be provided and
will specify the error, in which case the first argument is undefined.


%%%%%%%%%%%%%%%%%%%% GroupAtomicMod %%%%%%%%%%%%%%%%%%%%
\pagebreak
\subsubsection{\code{GroupAtomicMod}}
\label{api:Go:GroupAtomicMod}
\index{GroupAtomicMod!Go API}
Store the existing value modulo the specified number for each object in
\code{space} that matches \code{checks}.

This operation will only affect objects that match the provided \code{checks}.
Objects that do not match \code{checks} will be unaffected by the group call.
Each object that matches \code{checks} will be atomically updated with the check
on the object.  HyperDex guarantees that no object will be altered if the
\code{checks} do not pass at the time of the write.  Objects that are updated
concurrently with the group call may or may not be updated; however, regardless
of any other concurrent operations, the preceding guarantee will always hold.



\paragraph{Definition:}
\begin{gocode}
func (client *Client) GroupAtomicMod(spacename string, predicates []Predicate, attributes Attributes) (count uint64, err *Error)
\end{gocode}

\paragraph{Parameters:}
\begin{itemize}[noitemsep]
\item \code{spacename}\\
The name of the space as a C-string.

\item \code{predicates}\\
A set of predicates to check against.  \code{checks} is a map from the
attributes' names to the predicates to check.

\item \code{attributes}\\
The set of attributes to modify and their respective values.  \code{attrs}
points to an array of length \code{attrs\_sz}.

\end{itemize}

\paragraph{Returns:}
This function returns via the provided callback.  In the normal case, the first
argument will be the total number of objects that match the predicate.

If the operation encounters any error, the error argument will be provided and
will specify the error, in which case the first argument is undefined.


%%%%%%%%%%%%%%%%%%%% AtomicAnd %%%%%%%%%%%%%%%%%%%%
\pagebreak
\subsubsection{\code{AtomicAnd}}
\label{api:Go:AtomicAnd}
\index{AtomicAnd!Go API}
Store the bitwise AND of the existing value and the specified number for
each attribute.
This operation requires a pre-existing object in order to complete successfully.
If no object exists, the operation will fail with \code{NOTFOUND}.



\paragraph{Definition:}
\begin{gocode}
func (client *Client) AtomicAnd(spacename string, key Value, attributes Attributes) (err *Error)
\end{gocode}

\paragraph{Parameters:}
\begin{itemize}[noitemsep]
\item \code{spacename}\\
The name of the space as a C-string.

\item \code{key}\\
The key for the operation where \code{key} is a Javascript value.

\item \code{attributes}\\
The set of attributes to modify and their respective values.  \code{attrs}
points to an array of length \code{attrs\_sz}.

\end{itemize}

\paragraph{Returns:}
This function returns via the provided callback.  In the normal case, the first
argument will indicate success or failure of the operation with one of the
following values:

\begin{itemize}[noitemsep]
\item \code{true} if the operation succeeded
\item \code{false} if any provided predicates failed
\item \code{null} if the operation requires an existing value and none exist
\end{itemize}

If the operation encounters any error, the error argument will be provided and
will specify the error, in which case the first argument is undefined.


%%%%%%%%%%%%%%%%%%%% CondAtomicAnd %%%%%%%%%%%%%%%%%%%%
\pagebreak
\subsubsection{\code{CondAtomicAnd}}
\label{api:Go:CondAtomicAnd}
\index{CondAtomicAnd!Go API}
Store the bitwise AND of the existing value and the specified number for
each attribute if and only if the \code{checks} hold on the object.
This operation requires a pre-existing object in order to complete successfully.
If no object exists, the operation will fail with \code{NOTFOUND}.


This operation will succeed if and only if the predicates specified by
\code{checks} hold on the pre-existing object.  If any of the predicates are not
true for the existing object, then the operation will have no effect and fail
with \code{CMPFAIL}.

All checks are atomic with the write.  HyperDex guarantees that no other
operation will come between validating the checks, and writing the new version
of the object..



\paragraph{Definition:}
\begin{gocode}
func (client *Client) CondAtomicAnd(spacename string, key Value, predicates []Predicate, attributes Attributes) (err *Error)
\end{gocode}

\paragraph{Parameters:}
\begin{itemize}[noitemsep]
\item \code{spacename}\\
The name of the space as a C-string.

\item \code{key}\\
The key for the operation where \code{key} is a Javascript value.

\item \code{predicates}\\
A set of predicates to check against.  \code{checks} is a map from the
attributes' names to the predicates to check.

\item \code{attributes}\\
The set of attributes to modify and their respective values.  \code{attrs}
points to an array of length \code{attrs\_sz}.

\end{itemize}

\paragraph{Returns:}
This function returns via the provided callback.  In the normal case, the first
argument will indicate success or failure of the operation with one of the
following values:

\begin{itemize}[noitemsep]
\item \code{true} if the operation succeeded
\item \code{false} if any provided predicates failed
\item \code{null} if the operation requires an existing value and none exist
\end{itemize}

If the operation encounters any error, the error argument will be provided and
will specify the error, in which case the first argument is undefined.


%%%%%%%%%%%%%%%%%%%% GroupAtomicAnd %%%%%%%%%%%%%%%%%%%%
\pagebreak
\subsubsection{\code{GroupAtomicAnd}}
\label{api:Go:GroupAtomicAnd}
\index{GroupAtomicAnd!Go API}
Store the bitwise AND of the existing value and the specified number for
each object in \code{space} that matches \code{checks}.

This operation will only affect objects that match the provided \code{checks}.
Objects that do not match \code{checks} will be unaffected by the group call.
Each object that matches \code{checks} will be atomically updated with the check
on the object.  HyperDex guarantees that no object will be altered if the
\code{checks} do not pass at the time of the write.  Objects that are updated
concurrently with the group call may or may not be updated; however, regardless
of any other concurrent operations, the preceding guarantee will always hold.



\paragraph{Definition:}
\begin{gocode}
func (client *Client) GroupAtomicAnd(spacename string, predicates []Predicate, attributes Attributes) (count uint64, err *Error)
\end{gocode}

\paragraph{Parameters:}
\begin{itemize}[noitemsep]
\item \code{spacename}\\
The name of the space as a C-string.

\item \code{predicates}\\
A set of predicates to check against.  \code{checks} is a map from the
attributes' names to the predicates to check.

\item \code{attributes}\\
The set of attributes to modify and their respective values.  \code{attrs}
points to an array of length \code{attrs\_sz}.

\end{itemize}

\paragraph{Returns:}
This function returns via the provided callback.  In the normal case, the first
argument will be the total number of objects that match the predicate.

If the operation encounters any error, the error argument will be provided and
will specify the error, in which case the first argument is undefined.


%%%%%%%%%%%%%%%%%%%% AtomicOr %%%%%%%%%%%%%%%%%%%%
\pagebreak
\subsubsection{\code{AtomicOr}}
\label{api:Go:AtomicOr}
\index{AtomicOr!Go API}
Store the bitwise OR of the existing value and the specified number for each
attribute.
This operation requires a pre-existing object in order to complete successfully.
If no object exists, the operation will fail with \code{NOTFOUND}.



\paragraph{Definition:}
\begin{gocode}
func (client *Client) AtomicOr(spacename string, key Value, attributes Attributes) (err *Error)
\end{gocode}

\paragraph{Parameters:}
\begin{itemize}[noitemsep]
\item \code{spacename}\\
The name of the space as a C-string.

\item \code{key}\\
The key for the operation where \code{key} is a Javascript value.

\item \code{attributes}\\
The set of attributes to modify and their respective values.  \code{attrs}
points to an array of length \code{attrs\_sz}.

\end{itemize}

\paragraph{Returns:}
This function returns via the provided callback.  In the normal case, the first
argument will indicate success or failure of the operation with one of the
following values:

\begin{itemize}[noitemsep]
\item \code{true} if the operation succeeded
\item \code{false} if any provided predicates failed
\item \code{null} if the operation requires an existing value and none exist
\end{itemize}

If the operation encounters any error, the error argument will be provided and
will specify the error, in which case the first argument is undefined.


%%%%%%%%%%%%%%%%%%%% CondAtomicOr %%%%%%%%%%%%%%%%%%%%
\pagebreak
\subsubsection{\code{CondAtomicOr}}
\label{api:Go:CondAtomicOr}
\index{CondAtomicOr!Go API}
Store the bitwise OR of the existing value and the specified number for each
attribute if and only if the \code{checks} hold on the object.
This operation requires a pre-existing object in order to complete successfully.
If no object exists, the operation will fail with \code{NOTFOUND}.


This operation will succeed if and only if the predicates specified by
\code{checks} hold on the pre-existing object.  If any of the predicates are not
true for the existing object, then the operation will have no effect and fail
with \code{CMPFAIL}.

All checks are atomic with the write.  HyperDex guarantees that no other
operation will come between validating the checks, and writing the new version
of the object..



\paragraph{Definition:}
\begin{gocode}
func (client *Client) CondAtomicOr(spacename string, key Value, predicates []Predicate, attributes Attributes) (err *Error)
\end{gocode}

\paragraph{Parameters:}
\begin{itemize}[noitemsep]
\item \code{spacename}\\
The name of the space as a C-string.

\item \code{key}\\
The key for the operation where \code{key} is a Javascript value.

\item \code{predicates}\\
A set of predicates to check against.  \code{checks} is a map from the
attributes' names to the predicates to check.

\item \code{attributes}\\
The set of attributes to modify and their respective values.  \code{attrs}
points to an array of length \code{attrs\_sz}.

\end{itemize}

\paragraph{Returns:}
This function returns via the provided callback.  In the normal case, the first
argument will indicate success or failure of the operation with one of the
following values:

\begin{itemize}[noitemsep]
\item \code{true} if the operation succeeded
\item \code{false} if any provided predicates failed
\item \code{null} if the operation requires an existing value and none exist
\end{itemize}

If the operation encounters any error, the error argument will be provided and
will specify the error, in which case the first argument is undefined.


%%%%%%%%%%%%%%%%%%%% GroupAtomicOr %%%%%%%%%%%%%%%%%%%%
\pagebreak
\subsubsection{\code{GroupAtomicOr}}
\label{api:Go:GroupAtomicOr}
\index{GroupAtomicOr!Go API}
Store the bitwise OR of the existing value and the specified number for each
object in \code{space} that matches \code{checks}.

This operation will only affect objects that match the provided \code{checks}.
Objects that do not match \code{checks} will be unaffected by the group call.
Each object that matches \code{checks} will be atomically updated with the check
on the object.  HyperDex guarantees that no object will be altered if the
\code{checks} do not pass at the time of the write.  Objects that are updated
concurrently with the group call may or may not be updated; however, regardless
of any other concurrent operations, the preceding guarantee will always hold.



\paragraph{Definition:}
\begin{gocode}
func (client *Client) GroupAtomicOr(spacename string, predicates []Predicate, attributes Attributes) (count uint64, err *Error)
\end{gocode}

\paragraph{Parameters:}
\begin{itemize}[noitemsep]
\item \code{spacename}\\
The name of the space as a C-string.

\item \code{predicates}\\
A set of predicates to check against.  \code{checks} is a map from the
attributes' names to the predicates to check.

\item \code{attributes}\\
The set of attributes to modify and their respective values.  \code{attrs}
points to an array of length \code{attrs\_sz}.

\end{itemize}

\paragraph{Returns:}
This function returns via the provided callback.  In the normal case, the first
argument will be the total number of objects that match the predicate.

If the operation encounters any error, the error argument will be provided and
will specify the error, in which case the first argument is undefined.


%%%%%%%%%%%%%%%%%%%% AtomicXor %%%%%%%%%%%%%%%%%%%%
\pagebreak
\subsubsection{\code{AtomicXor}}
\label{api:Go:AtomicXor}
\index{AtomicXor!Go API}
Store the bitwise XOR of the existing value and the specified number for each
attribute.
This operation requires a pre-existing object in order to complete successfully.
If no object exists, the operation will fail with \code{NOTFOUND}.



\paragraph{Definition:}
\begin{gocode}
func (client *Client) AtomicXor(spacename string, key Value, attributes Attributes) (err *Error)
\end{gocode}

\paragraph{Parameters:}
\begin{itemize}[noitemsep]
\item \code{spacename}\\
The name of the space as a C-string.

\item \code{key}\\
The key for the operation where \code{key} is a Javascript value.

\item \code{attributes}\\
The set of attributes to modify and their respective values.  \code{attrs}
points to an array of length \code{attrs\_sz}.

\end{itemize}

\paragraph{Returns:}
This function returns via the provided callback.  In the normal case, the first
argument will indicate success or failure of the operation with one of the
following values:

\begin{itemize}[noitemsep]
\item \code{true} if the operation succeeded
\item \code{false} if any provided predicates failed
\item \code{null} if the operation requires an existing value and none exist
\end{itemize}

If the operation encounters any error, the error argument will be provided and
will specify the error, in which case the first argument is undefined.


%%%%%%%%%%%%%%%%%%%% CondAtomicXor %%%%%%%%%%%%%%%%%%%%
\pagebreak
\subsubsection{\code{CondAtomicXor}}
\label{api:Go:CondAtomicXor}
\index{CondAtomicXor!Go API}
Store the bitwise XOR of the existing value and the specified number for each
attribute if and only if \code{checks} hold on the object.
This operation requires a pre-existing object in order to complete successfully.
If no object exists, the operation will fail with \code{NOTFOUND}.


This operation will succeed if and only if the predicates specified by
\code{checks} hold on the pre-existing object.  If any of the predicates are not
true for the existing object, then the operation will have no effect and fail
with \code{CMPFAIL}.

All checks are atomic with the write.  HyperDex guarantees that no other
operation will come between validating the checks, and writing the new version
of the object..



\paragraph{Definition:}
\begin{gocode}
func (client *Client) CondAtomicXor(spacename string, key Value, predicates []Predicate, attributes Attributes) (err *Error)
\end{gocode}

\paragraph{Parameters:}
\begin{itemize}[noitemsep]
\item \code{spacename}\\
The name of the space as a C-string.

\item \code{key}\\
The key for the operation where \code{key} is a Javascript value.

\item \code{predicates}\\
A set of predicates to check against.  \code{checks} is a map from the
attributes' names to the predicates to check.

\item \code{attributes}\\
The set of attributes to modify and their respective values.  \code{attrs}
points to an array of length \code{attrs\_sz}.

\end{itemize}

\paragraph{Returns:}
This function returns via the provided callback.  In the normal case, the first
argument will indicate success or failure of the operation with one of the
following values:

\begin{itemize}[noitemsep]
\item \code{true} if the operation succeeded
\item \code{false} if any provided predicates failed
\item \code{null} if the operation requires an existing value and none exist
\end{itemize}

If the operation encounters any error, the error argument will be provided and
will specify the error, in which case the first argument is undefined.


%%%%%%%%%%%%%%%%%%%% GroupAtomicXor %%%%%%%%%%%%%%%%%%%%
\pagebreak
\subsubsection{\code{GroupAtomicXor}}
\label{api:Go:GroupAtomicXor}
\index{GroupAtomicXor!Go API}
Store the bitwise XOR of the existing value and the specified number for each
object in \code{space} that matches \code{checks}.

This operation will only affect objects that match the provided \code{checks}.
Objects that do not match \code{checks} will be unaffected by the group call.
Each object that matches \code{checks} will be atomically updated with the check
on the object.  HyperDex guarantees that no object will be altered if the
\code{checks} do not pass at the time of the write.  Objects that are updated
concurrently with the group call may or may not be updated; however, regardless
of any other concurrent operations, the preceding guarantee will always hold.



\paragraph{Definition:}
\begin{gocode}
func (client *Client) GroupAtomicXor(spacename string, predicates []Predicate, attributes Attributes) (count uint64, err *Error)
\end{gocode}

\paragraph{Parameters:}
\begin{itemize}[noitemsep]
\item \code{spacename}\\
The name of the space as a C-string.

\item \code{predicates}\\
A set of predicates to check against.  \code{checks} is a map from the
attributes' names to the predicates to check.

\item \code{attributes}\\
The set of attributes to modify and their respective values.  \code{attrs}
points to an array of length \code{attrs\_sz}.

\end{itemize}

\paragraph{Returns:}
This function returns via the provided callback.  In the normal case, the first
argument will be the total number of objects that match the predicate.

If the operation encounters any error, the error argument will be provided and
will specify the error, in which case the first argument is undefined.


%%%%%%%%%%%%%%%%%%%% AtomicMin %%%%%%%%%%%%%%%%%%%%
\pagebreak
\subsubsection{\code{AtomicMin}}
\label{api:Go:AtomicMin}
\index{AtomicMin!Go API}
Store the minimum of the existing value and the provided value for each
attribute.
This operation requires a pre-existing object in order to complete successfully.
If no object exists, the operation will fail with \code{NOTFOUND}.



\paragraph{Definition:}
\begin{gocode}
func (client *Client) AtomicMin(spacename string, key Value, attributes Attributes) (err *Error)
\end{gocode}

\paragraph{Parameters:}
\begin{itemize}[noitemsep]
\item \code{spacename}\\
The name of the space as a C-string.

\item \code{key}\\
The key for the operation where \code{key} is a Javascript value.

\item \code{attributes}\\
The set of attributes to modify and their respective values.  \code{attrs}
points to an array of length \code{attrs\_sz}.

\end{itemize}

\paragraph{Returns:}
This function returns via the provided callback.  In the normal case, the first
argument will indicate success or failure of the operation with one of the
following values:

\begin{itemize}[noitemsep]
\item \code{true} if the operation succeeded
\item \code{false} if any provided predicates failed
\item \code{null} if the operation requires an existing value and none exist
\end{itemize}

If the operation encounters any error, the error argument will be provided and
will specify the error, in which case the first argument is undefined.


%%%%%%%%%%%%%%%%%%%% CondAtomicMin %%%%%%%%%%%%%%%%%%%%
\pagebreak
\subsubsection{\code{CondAtomicMin}}
\label{api:Go:CondAtomicMin}
\index{CondAtomicMin!Go API}
Store the minimum of the existing value and the provided value for each
attribute if and only if \code{checks} hold on the object.
This operation requires a pre-existing object in order to complete successfully.
If no object exists, the operation will fail with \code{NOTFOUND}.


This operation will succeed if and only if the predicates specified by
\code{checks} hold on the pre-existing object.  If any of the predicates are not
true for the existing object, then the operation will have no effect and fail
with \code{CMPFAIL}.

All checks are atomic with the write.  HyperDex guarantees that no other
operation will come between validating the checks, and writing the new version
of the object..



\paragraph{Definition:}
\begin{gocode}
func (client *Client) CondAtomicMin(spacename string, key Value, predicates []Predicate, attributes Attributes) (err *Error)
\end{gocode}

\paragraph{Parameters:}
\begin{itemize}[noitemsep]
\item \code{spacename}\\
The name of the space as a C-string.

\item \code{key}\\
The key for the operation where \code{key} is a Javascript value.

\item \code{predicates}\\
A set of predicates to check against.  \code{checks} is a map from the
attributes' names to the predicates to check.

\item \code{attributes}\\
The set of attributes to modify and their respective values.  \code{attrs}
points to an array of length \code{attrs\_sz}.

\end{itemize}

\paragraph{Returns:}
This function returns via the provided callback.  In the normal case, the first
argument will indicate success or failure of the operation with one of the
following values:

\begin{itemize}[noitemsep]
\item \code{true} if the operation succeeded
\item \code{false} if any provided predicates failed
\item \code{null} if the operation requires an existing value and none exist
\end{itemize}

If the operation encounters any error, the error argument will be provided and
will specify the error, in which case the first argument is undefined.


%%%%%%%%%%%%%%%%%%%% GroupAtomicMin %%%%%%%%%%%%%%%%%%%%
\pagebreak
\subsubsection{\code{GroupAtomicMin}}
\label{api:Go:GroupAtomicMin}
\index{GroupAtomicMin!Go API}
Store the minimum of the existing value and the provided value for each
object in \code{space} that matches \code{checks}.

This operation will only affect objects that match the provided \code{checks}.
Objects that do not match \code{checks} will be unaffected by the group call.
Each object that matches \code{checks} will be atomically updated with the check
on the object.  HyperDex guarantees that no object will be altered if the
\code{checks} do not pass at the time of the write.  Objects that are updated
concurrently with the group call may or may not be updated; however, regardless
of any other concurrent operations, the preceding guarantee will always hold.



\paragraph{Definition:}
\begin{gocode}
func (client *Client) GroupAtomicMin(spacename string, predicates []Predicate, attributes Attributes) (count uint64, err *Error)
\end{gocode}

\paragraph{Parameters:}
\begin{itemize}[noitemsep]
\item \code{spacename}\\
The name of the space as a C-string.

\item \code{predicates}\\
A set of predicates to check against.  \code{checks} is a map from the
attributes' names to the predicates to check.

\item \code{attributes}\\
The set of attributes to modify and their respective values.  \code{attrs}
points to an array of length \code{attrs\_sz}.

\end{itemize}

\paragraph{Returns:}
This function returns via the provided callback.  In the normal case, the first
argument will be the total number of objects that match the predicate.

If the operation encounters any error, the error argument will be provided and
will specify the error, in which case the first argument is undefined.


%%%%%%%%%%%%%%%%%%%% AtomicMax %%%%%%%%%%%%%%%%%%%%
\pagebreak
\subsubsection{\code{AtomicMax}}
\label{api:Go:AtomicMax}
\index{AtomicMax!Go API}
Store the maximum of the existing value and the provided value for each
attribute.
This operation requires a pre-existing object in order to complete successfully.
If no object exists, the operation will fail with \code{NOTFOUND}.



\paragraph{Definition:}
\begin{gocode}
func (client *Client) AtomicMax(spacename string, key Value, attributes Attributes) (err *Error)
\end{gocode}

\paragraph{Parameters:}
\begin{itemize}[noitemsep]
\item \code{spacename}\\
The name of the space as a C-string.

\item \code{key}\\
The key for the operation where \code{key} is a Javascript value.

\item \code{attributes}\\
The set of attributes to modify and their respective values.  \code{attrs}
points to an array of length \code{attrs\_sz}.

\end{itemize}

\paragraph{Returns:}
This function returns via the provided callback.  In the normal case, the first
argument will indicate success or failure of the operation with one of the
following values:

\begin{itemize}[noitemsep]
\item \code{true} if the operation succeeded
\item \code{false} if any provided predicates failed
\item \code{null} if the operation requires an existing value and none exist
\end{itemize}

If the operation encounters any error, the error argument will be provided and
will specify the error, in which case the first argument is undefined.


%%%%%%%%%%%%%%%%%%%% CondAtomicMax %%%%%%%%%%%%%%%%%%%%
\pagebreak
\subsubsection{\code{CondAtomicMax}}
\label{api:Go:CondAtomicMax}
\index{CondAtomicMax!Go API}
Store the maximum of the existing value and the provided value for each
attribute if and only if \code{checks} hold on the object.
This operation requires a pre-existing object in order to complete successfully.
If no object exists, the operation will fail with \code{NOTFOUND}.


This operation will succeed if and only if the predicates specified by
\code{checks} hold on the pre-existing object.  If any of the predicates are not
true for the existing object, then the operation will have no effect and fail
with \code{CMPFAIL}.

All checks are atomic with the write.  HyperDex guarantees that no other
operation will come between validating the checks, and writing the new version
of the object..



\paragraph{Definition:}
\begin{gocode}
func (client *Client) CondAtomicMax(spacename string, key Value, predicates []Predicate, attributes Attributes) (err *Error)
\end{gocode}

\paragraph{Parameters:}
\begin{itemize}[noitemsep]
\item \code{spacename}\\
The name of the space as a C-string.

\item \code{key}\\
The key for the operation where \code{key} is a Javascript value.

\item \code{predicates}\\
A set of predicates to check against.  \code{checks} is a map from the
attributes' names to the predicates to check.

\item \code{attributes}\\
The set of attributes to modify and their respective values.  \code{attrs}
points to an array of length \code{attrs\_sz}.

\end{itemize}

\paragraph{Returns:}
This function returns via the provided callback.  In the normal case, the first
argument will indicate success or failure of the operation with one of the
following values:

\begin{itemize}[noitemsep]
\item \code{true} if the operation succeeded
\item \code{false} if any provided predicates failed
\item \code{null} if the operation requires an existing value and none exist
\end{itemize}

If the operation encounters any error, the error argument will be provided and
will specify the error, in which case the first argument is undefined.


%%%%%%%%%%%%%%%%%%%% GroupAtomicMax %%%%%%%%%%%%%%%%%%%%
\pagebreak
\subsubsection{\code{GroupAtomicMax}}
\label{api:Go:GroupAtomicMax}
\index{GroupAtomicMax!Go API}
Store the maximum of the existing value and the provided value for each
object in \code{space} that matches \code{checks}.

This operation will only affect objects that match the provided \code{checks}.
Objects that do not match \code{checks} will be unaffected by the group call.
Each object that matches \code{checks} will be atomically updated with the check
on the object.  HyperDex guarantees that no object will be altered if the
\code{checks} do not pass at the time of the write.  Objects that are updated
concurrently with the group call may or may not be updated; however, regardless
of any other concurrent operations, the preceding guarantee will always hold.



\paragraph{Definition:}
\begin{gocode}
func (client *Client) GroupAtomicMax(spacename string, predicates []Predicate, attributes Attributes) (count uint64, err *Error)
\end{gocode}

\paragraph{Parameters:}
\begin{itemize}[noitemsep]
\item \code{spacename}\\
The name of the space as a C-string.

\item \code{predicates}\\
A set of predicates to check against.  \code{checks} is a map from the
attributes' names to the predicates to check.

\item \code{attributes}\\
The set of attributes to modify and their respective values.  \code{attrs}
points to an array of length \code{attrs\_sz}.

\end{itemize}

\paragraph{Returns:}
This function returns via the provided callback.  In the normal case, the first
argument will be the total number of objects that match the predicate.

If the operation encounters any error, the error argument will be provided and
will specify the error, in which case the first argument is undefined.


%%%%%%%%%%%%%%%%%%%% StringPrepend %%%%%%%%%%%%%%%%%%%%
\pagebreak
\subsubsection{\code{StringPrepend}}
\label{api:Go:StringPrepend}
\index{StringPrepend!Go API}
Prepend the specified string to the existing value for each attribute.
This operation requires a pre-existing object in order to complete successfully.
If no object exists, the operation will fail with \code{NOTFOUND}.



\paragraph{Definition:}
\begin{gocode}
func (client *Client) StringPrepend(spacename string, key Value, attributes Attributes) (err *Error)
\end{gocode}

\paragraph{Parameters:}
\begin{itemize}[noitemsep]
\item \code{spacename}\\
The name of the space as a C-string.

\item \code{key}\\
The key for the operation where \code{key} is a Javascript value.

\item \code{attributes}\\
The set of attributes to modify and their respective values.  \code{attrs}
points to an array of length \code{attrs\_sz}.

\end{itemize}

\paragraph{Returns:}
This function returns via the provided callback.  In the normal case, the first
argument will indicate success or failure of the operation with one of the
following values:

\begin{itemize}[noitemsep]
\item \code{true} if the operation succeeded
\item \code{false} if any provided predicates failed
\item \code{null} if the operation requires an existing value and none exist
\end{itemize}

If the operation encounters any error, the error argument will be provided and
will specify the error, in which case the first argument is undefined.


%%%%%%%%%%%%%%%%%%%% CondStringPrepend %%%%%%%%%%%%%%%%%%%%
\pagebreak
\subsubsection{\code{CondStringPrepend}}
\label{api:Go:CondStringPrepend}
\index{CondStringPrepend!Go API}
Prepend the specified string to the existing value for each attribute if and
only if the \code{checks} hold on the object.
This operation requires a pre-existing object in order to complete successfully.
If no object exists, the operation will fail with \code{NOTFOUND}.


This operation will succeed if and only if the predicates specified by
\code{checks} hold on the pre-existing object.  If any of the predicates are not
true for the existing object, then the operation will have no effect and fail
with \code{CMPFAIL}.

All checks are atomic with the write.  HyperDex guarantees that no other
operation will come between validating the checks, and writing the new version
of the object..



\paragraph{Definition:}
\begin{gocode}
func (client *Client) CondStringPrepend(spacename string, key Value, predicates []Predicate, attributes Attributes) (err *Error)
\end{gocode}

\paragraph{Parameters:}
\begin{itemize}[noitemsep]
\item \code{spacename}\\
The name of the space as a C-string.

\item \code{key}\\
The key for the operation where \code{key} is a Javascript value.

\item \code{predicates}\\
A set of predicates to check against.  \code{checks} is a map from the
attributes' names to the predicates to check.

\item \code{attributes}\\
The set of attributes to modify and their respective values.  \code{attrs}
points to an array of length \code{attrs\_sz}.

\end{itemize}

\paragraph{Returns:}
This function returns via the provided callback.  In the normal case, the first
argument will indicate success or failure of the operation with one of the
following values:

\begin{itemize}[noitemsep]
\item \code{true} if the operation succeeded
\item \code{false} if any provided predicates failed
\item \code{null} if the operation requires an existing value and none exist
\end{itemize}

If the operation encounters any error, the error argument will be provided and
will specify the error, in which case the first argument is undefined.


%%%%%%%%%%%%%%%%%%%% GroupStringPrepend %%%%%%%%%%%%%%%%%%%%
\pagebreak
\subsubsection{\code{GroupStringPrepend}}
\label{api:Go:GroupStringPrepend}
\index{GroupStringPrepend!Go API}
Prepend the specified string to the existing value for each object in
\code{space} that matches \code{checks}.

This operation will only affect objects that match the provided \code{checks}.
Objects that do not match \code{checks} will be unaffected by the group call.
Each object that matches \code{checks} will be atomically updated with the check
on the object.  HyperDex guarantees that no object will be altered if the
\code{checks} do not pass at the time of the write.  Objects that are updated
concurrently with the group call may or may not be updated; however, regardless
of any other concurrent operations, the preceding guarantee will always hold.



\paragraph{Definition:}
\begin{gocode}
func (client *Client) GroupStringPrepend(spacename string, predicates []Predicate, attributes Attributes) (count uint64, err *Error)
\end{gocode}

\paragraph{Parameters:}
\begin{itemize}[noitemsep]
\item \code{spacename}\\
The name of the space as a C-string.

\item \code{predicates}\\
A set of predicates to check against.  \code{checks} is a map from the
attributes' names to the predicates to check.

\item \code{attributes}\\
The set of attributes to modify and their respective values.  \code{attrs}
points to an array of length \code{attrs\_sz}.

\end{itemize}

\paragraph{Returns:}
This function returns via the provided callback.  In the normal case, the first
argument will be the total number of objects that match the predicate.

If the operation encounters any error, the error argument will be provided and
will specify the error, in which case the first argument is undefined.


%%%%%%%%%%%%%%%%%%%% StringAppend %%%%%%%%%%%%%%%%%%%%
\pagebreak
\subsubsection{\code{StringAppend}}
\label{api:Go:StringAppend}
\index{StringAppend!Go API}
Append the specified string to the existing value for each attribute.
This operation requires a pre-existing object in order to complete successfully.
If no object exists, the operation will fail with \code{NOTFOUND}.



\paragraph{Definition:}
\begin{gocode}
func (client *Client) StringAppend(spacename string, key Value, attributes Attributes) (err *Error)
\end{gocode}

\paragraph{Parameters:}
\begin{itemize}[noitemsep]
\item \code{spacename}\\
The name of the space as a C-string.

\item \code{key}\\
The key for the operation where \code{key} is a Javascript value.

\item \code{attributes}\\
The set of attributes to modify and their respective values.  \code{attrs}
points to an array of length \code{attrs\_sz}.

\end{itemize}

\paragraph{Returns:}
This function returns via the provided callback.  In the normal case, the first
argument will indicate success or failure of the operation with one of the
following values:

\begin{itemize}[noitemsep]
\item \code{true} if the operation succeeded
\item \code{false} if any provided predicates failed
\item \code{null} if the operation requires an existing value and none exist
\end{itemize}

If the operation encounters any error, the error argument will be provided and
will specify the error, in which case the first argument is undefined.


%%%%%%%%%%%%%%%%%%%% CondStringAppend %%%%%%%%%%%%%%%%%%%%
\pagebreak
\subsubsection{\code{CondStringAppend}}
\label{api:Go:CondStringAppend}
\index{CondStringAppend!Go API}
Append the specified string to the existing value for each attribute if and only
if \code{checks} hold on the object.
This operation requires a pre-existing object in order to complete successfully.
If no object exists, the operation will fail with \code{NOTFOUND}.


This operation will succeed if and only if the predicates specified by
\code{checks} hold on the pre-existing object.  If any of the predicates are not
true for the existing object, then the operation will have no effect and fail
with \code{CMPFAIL}.

All checks are atomic with the write.  HyperDex guarantees that no other
operation will come between validating the checks, and writing the new version
of the object..



\paragraph{Definition:}
\begin{gocode}
func (client *Client) CondStringAppend(spacename string, key Value, predicates []Predicate, attributes Attributes) (err *Error)
\end{gocode}

\paragraph{Parameters:}
\begin{itemize}[noitemsep]
\item \code{spacename}\\
The name of the space as a C-string.

\item \code{key}\\
The key for the operation where \code{key} is a Javascript value.

\item \code{predicates}\\
A set of predicates to check against.  \code{checks} is a map from the
attributes' names to the predicates to check.

\item \code{attributes}\\
The set of attributes to modify and their respective values.  \code{attrs}
points to an array of length \code{attrs\_sz}.

\end{itemize}

\paragraph{Returns:}
This function returns via the provided callback.  In the normal case, the first
argument will indicate success or failure of the operation with one of the
following values:

\begin{itemize}[noitemsep]
\item \code{true} if the operation succeeded
\item \code{false} if any provided predicates failed
\item \code{null} if the operation requires an existing value and none exist
\end{itemize}

If the operation encounters any error, the error argument will be provided and
will specify the error, in which case the first argument is undefined.


%%%%%%%%%%%%%%%%%%%% GroupStringAppend %%%%%%%%%%%%%%%%%%%%
\pagebreak
\subsubsection{\code{GroupStringAppend}}
\label{api:Go:GroupStringAppend}
\index{GroupStringAppend!Go API}
Append the specified string to the existing value for each object in
\code{space} that matches \code{checks}.

This operation will only affect objects that match the provided \code{checks}.
Objects that do not match \code{checks} will be unaffected by the group call.
Each object that matches \code{checks} will be atomically updated with the check
on the object.  HyperDex guarantees that no object will be altered if the
\code{checks} do not pass at the time of the write.  Objects that are updated
concurrently with the group call may or may not be updated; however, regardless
of any other concurrent operations, the preceding guarantee will always hold.



\paragraph{Definition:}
\begin{gocode}
func (client *Client) GroupStringAppend(spacename string, predicates []Predicate, attributes Attributes) (count uint64, err *Error)
\end{gocode}

\paragraph{Parameters:}
\begin{itemize}[noitemsep]
\item \code{spacename}\\
The name of the space as a C-string.

\item \code{predicates}\\
A set of predicates to check against.  \code{checks} is a map from the
attributes' names to the predicates to check.

\item \code{attributes}\\
The set of attributes to modify and their respective values.  \code{attrs}
points to an array of length \code{attrs\_sz}.

\end{itemize}

\paragraph{Returns:}
This function returns via the provided callback.  In the normal case, the first
argument will be the total number of objects that match the predicate.

If the operation encounters any error, the error argument will be provided and
will specify the error, in which case the first argument is undefined.


%%%%%%%%%%%%%%%%%%%% ListLpush %%%%%%%%%%%%%%%%%%%%
\pagebreak
\subsubsection{\code{ListLpush}}
\label{api:Go:ListLpush}
\index{ListLpush!Go API}
Push the specified value onto the front of the list for each attribute.
This operation requires a pre-existing object in order to complete successfully.
If no object exists, the operation will fail with \code{NOTFOUND}.



\paragraph{Definition:}
\begin{gocode}
func (client *Client) ListLpush(spacename string, key Value, attributes Attributes) (err *Error)
\end{gocode}

\paragraph{Parameters:}
\begin{itemize}[noitemsep]
\item \code{spacename}\\
The name of the space as a C-string.

\item \code{key}\\
The key for the operation where \code{key} is a Javascript value.

\item \code{attributes}\\
The set of attributes to modify and their respective values.  \code{attrs}
points to an array of length \code{attrs\_sz}.

\end{itemize}

\paragraph{Returns:}
This function returns via the provided callback.  In the normal case, the first
argument will indicate success or failure of the operation with one of the
following values:

\begin{itemize}[noitemsep]
\item \code{true} if the operation succeeded
\item \code{false} if any provided predicates failed
\item \code{null} if the operation requires an existing value and none exist
\end{itemize}

If the operation encounters any error, the error argument will be provided and
will specify the error, in which case the first argument is undefined.


%%%%%%%%%%%%%%%%%%%% CondListLpush %%%%%%%%%%%%%%%%%%%%
\pagebreak
\subsubsection{\code{CondListLpush}}
\label{api:Go:CondListLpush}
\index{CondListLpush!Go API}
Push the specified value onto the front of the list for each attribute if and
only if \code{checks} hold on the object.
This operation requires a pre-existing object in order to complete successfully.
If no object exists, the operation will fail with \code{NOTFOUND}.


This operation will succeed if and only if the predicates specified by
\code{checks} hold on the pre-existing object.  If any of the predicates are not
true for the existing object, then the operation will have no effect and fail
with \code{CMPFAIL}.

All checks are atomic with the write.  HyperDex guarantees that no other
operation will come between validating the checks, and writing the new version
of the object..



\paragraph{Definition:}
\begin{gocode}
func (client *Client) CondListLpush(spacename string, key Value, predicates []Predicate, attributes Attributes) (err *Error)
\end{gocode}

\paragraph{Parameters:}
\begin{itemize}[noitemsep]
\item \code{spacename}\\
The name of the space as a C-string.

\item \code{key}\\
The key for the operation where \code{key} is a Javascript value.

\item \code{predicates}\\
A set of predicates to check against.  \code{checks} is a map from the
attributes' names to the predicates to check.

\item \code{attributes}\\
The set of attributes to modify and their respective values.  \code{attrs}
points to an array of length \code{attrs\_sz}.

\end{itemize}

\paragraph{Returns:}
This function returns via the provided callback.  In the normal case, the first
argument will indicate success or failure of the operation with one of the
following values:

\begin{itemize}[noitemsep]
\item \code{true} if the operation succeeded
\item \code{false} if any provided predicates failed
\item \code{null} if the operation requires an existing value and none exist
\end{itemize}

If the operation encounters any error, the error argument will be provided and
will specify the error, in which case the first argument is undefined.


%%%%%%%%%%%%%%%%%%%% GroupListLpush %%%%%%%%%%%%%%%%%%%%
\pagebreak
\subsubsection{\code{GroupListLpush}}
\label{api:Go:GroupListLpush}
\index{GroupListLpush!Go API}
Push the specified value onto the front of the list for each object in
\code{space} that matches \code{checks}.

This operation will only affect objects that match the provided \code{checks}.
Objects that do not match \code{checks} will be unaffected by the group call.
Each object that matches \code{checks} will be atomically updated with the check
on the object.  HyperDex guarantees that no object will be altered if the
\code{checks} do not pass at the time of the write.  Objects that are updated
concurrently with the group call may or may not be updated; however, regardless
of any other concurrent operations, the preceding guarantee will always hold.



\paragraph{Definition:}
\begin{gocode}
func (client *Client) GroupListLpush(spacename string, predicates []Predicate, attributes Attributes) (count uint64, err *Error)
\end{gocode}

\paragraph{Parameters:}
\begin{itemize}[noitemsep]
\item \code{spacename}\\
The name of the space as a C-string.

\item \code{predicates}\\
A set of predicates to check against.  \code{checks} is a map from the
attributes' names to the predicates to check.

\item \code{attributes}\\
The set of attributes to modify and their respective values.  \code{attrs}
points to an array of length \code{attrs\_sz}.

\end{itemize}

\paragraph{Returns:}
This function returns via the provided callback.  In the normal case, the first
argument will be the total number of objects that match the predicate.

If the operation encounters any error, the error argument will be provided and
will specify the error, in which case the first argument is undefined.


%%%%%%%%%%%%%%%%%%%% ListRpush %%%%%%%%%%%%%%%%%%%%
\pagebreak
\subsubsection{\code{ListRpush}}
\label{api:Go:ListRpush}
\index{ListRpush!Go API}
Push the specified value onto the back of the list for each attribute.
This operation requires a pre-existing object in order to complete successfully.
If no object exists, the operation will fail with \code{NOTFOUND}.



\paragraph{Definition:}
\begin{gocode}
func (client *Client) ListRpush(spacename string, key Value, attributes Attributes) (err *Error)
\end{gocode}

\paragraph{Parameters:}
\begin{itemize}[noitemsep]
\item \code{spacename}\\
The name of the space as a C-string.

\item \code{key}\\
The key for the operation where \code{key} is a Javascript value.

\item \code{attributes}\\
The set of attributes to modify and their respective values.  \code{attrs}
points to an array of length \code{attrs\_sz}.

\end{itemize}

\paragraph{Returns:}
This function returns via the provided callback.  In the normal case, the first
argument will indicate success or failure of the operation with one of the
following values:

\begin{itemize}[noitemsep]
\item \code{true} if the operation succeeded
\item \code{false} if any provided predicates failed
\item \code{null} if the operation requires an existing value and none exist
\end{itemize}

If the operation encounters any error, the error argument will be provided and
will specify the error, in which case the first argument is undefined.


%%%%%%%%%%%%%%%%%%%% CondListRpush %%%%%%%%%%%%%%%%%%%%
\pagebreak
\subsubsection{\code{CondListRpush}}
\label{api:Go:CondListRpush}
\index{CondListRpush!Go API}
Push the specified value onto the back of the list for each attribute if and
only if the \code{checks} hold on the object.
This operation requires a pre-existing object in order to complete successfully.
If no object exists, the operation will fail with \code{NOTFOUND}.


This operation will succeed if and only if the predicates specified by
\code{checks} hold on the pre-existing object.  If any of the predicates are not
true for the existing object, then the operation will have no effect and fail
with \code{CMPFAIL}.

All checks are atomic with the write.  HyperDex guarantees that no other
operation will come between validating the checks, and writing the new version
of the object..



\paragraph{Definition:}
\begin{gocode}
func (client *Client) CondListRpush(spacename string, key Value, predicates []Predicate, attributes Attributes) (err *Error)
\end{gocode}

\paragraph{Parameters:}
\begin{itemize}[noitemsep]
\item \code{spacename}\\
The name of the space as a C-string.

\item \code{key}\\
The key for the operation where \code{key} is a Javascript value.

\item \code{predicates}\\
A set of predicates to check against.  \code{checks} is a map from the
attributes' names to the predicates to check.

\item \code{attributes}\\
The set of attributes to modify and their respective values.  \code{attrs}
points to an array of length \code{attrs\_sz}.

\end{itemize}

\paragraph{Returns:}
This function returns via the provided callback.  In the normal case, the first
argument will indicate success or failure of the operation with one of the
following values:

\begin{itemize}[noitemsep]
\item \code{true} if the operation succeeded
\item \code{false} if any provided predicates failed
\item \code{null} if the operation requires an existing value and none exist
\end{itemize}

If the operation encounters any error, the error argument will be provided and
will specify the error, in which case the first argument is undefined.


%%%%%%%%%%%%%%%%%%%% GroupListRpush %%%%%%%%%%%%%%%%%%%%
\pagebreak
\subsubsection{\code{GroupListRpush}}
\label{api:Go:GroupListRpush}
\index{GroupListRpush!Go API}
Push the specified value onto the back of the list for each object in
\code{space} that matches \code{checks}.

This operation will only affect objects that match the provided \code{checks}.
Objects that do not match \code{checks} will be unaffected by the group call.
Each object that matches \code{checks} will be atomically updated with the check
on the object.  HyperDex guarantees that no object will be altered if the
\code{checks} do not pass at the time of the write.  Objects that are updated
concurrently with the group call may or may not be updated; however, regardless
of any other concurrent operations, the preceding guarantee will always hold.



\paragraph{Definition:}
\begin{gocode}
func (client *Client) GroupListRpush(spacename string, predicates []Predicate, attributes Attributes) (count uint64, err *Error)
\end{gocode}

\paragraph{Parameters:}
\begin{itemize}[noitemsep]
\item \code{spacename}\\
The name of the space as a C-string.

\item \code{predicates}\\
A set of predicates to check against.  \code{checks} is a map from the
attributes' names to the predicates to check.

\item \code{attributes}\\
The set of attributes to modify and their respective values.  \code{attrs}
points to an array of length \code{attrs\_sz}.

\end{itemize}

\paragraph{Returns:}
This function returns via the provided callback.  In the normal case, the first
argument will be the total number of objects that match the predicate.

If the operation encounters any error, the error argument will be provided and
will specify the error, in which case the first argument is undefined.


%%%%%%%%%%%%%%%%%%%% SetAdd %%%%%%%%%%%%%%%%%%%%
\pagebreak
\subsubsection{\code{SetAdd}}
\label{api:Go:SetAdd}
\index{SetAdd!Go API}
Add the specified value to the set for each attribute.
This operation requires a pre-existing object in order to complete successfully.
If no object exists, the operation will fail with \code{NOTFOUND}.



\paragraph{Definition:}
\begin{gocode}
func (client *Client) SetAdd(spacename string, key Value, attributes Attributes) (err *Error)
\end{gocode}

\paragraph{Parameters:}
\begin{itemize}[noitemsep]
\item \code{spacename}\\
The name of the space as a C-string.

\item \code{key}\\
The key for the operation where \code{key} is a Javascript value.

\item \code{attributes}\\
The set of attributes to modify and their respective values.  \code{attrs}
points to an array of length \code{attrs\_sz}.

\end{itemize}

\paragraph{Returns:}
This function returns via the provided callback.  In the normal case, the first
argument will indicate success or failure of the operation with one of the
following values:

\begin{itemize}[noitemsep]
\item \code{true} if the operation succeeded
\item \code{false} if any provided predicates failed
\item \code{null} if the operation requires an existing value and none exist
\end{itemize}

If the operation encounters any error, the error argument will be provided and
will specify the error, in which case the first argument is undefined.


%%%%%%%%%%%%%%%%%%%% CondSetAdd %%%%%%%%%%%%%%%%%%%%
\pagebreak
\subsubsection{\code{CondSetAdd}}
\label{api:Go:CondSetAdd}
\index{CondSetAdd!Go API}
Add the specified value to the set for each attribute if and only if the
\code{checks} hold on the object.
This operation requires a pre-existing object in order to complete successfully.
If no object exists, the operation will fail with \code{NOTFOUND}.


This operation will succeed if and only if the predicates specified by
\code{checks} hold on the pre-existing object.  If any of the predicates are not
true for the existing object, then the operation will have no effect and fail
with \code{CMPFAIL}.

All checks are atomic with the write.  HyperDex guarantees that no other
operation will come between validating the checks, and writing the new version
of the object..



\paragraph{Definition:}
\begin{gocode}
func (client *Client) CondSetAdd(spacename string, key Value, predicates []Predicate, attributes Attributes) (err *Error)
\end{gocode}

\paragraph{Parameters:}
\begin{itemize}[noitemsep]
\item \code{spacename}\\
The name of the space as a C-string.

\item \code{key}\\
The key for the operation where \code{key} is a Javascript value.

\item \code{predicates}\\
A set of predicates to check against.  \code{checks} is a map from the
attributes' names to the predicates to check.

\item \code{attributes}\\
The set of attributes to modify and their respective values.  \code{attrs}
points to an array of length \code{attrs\_sz}.

\end{itemize}

\paragraph{Returns:}
This function returns via the provided callback.  In the normal case, the first
argument will indicate success or failure of the operation with one of the
following values:

\begin{itemize}[noitemsep]
\item \code{true} if the operation succeeded
\item \code{false} if any provided predicates failed
\item \code{null} if the operation requires an existing value and none exist
\end{itemize}

If the operation encounters any error, the error argument will be provided and
will specify the error, in which case the first argument is undefined.


%%%%%%%%%%%%%%%%%%%% GroupSetAdd %%%%%%%%%%%%%%%%%%%%
\pagebreak
\subsubsection{\code{GroupSetAdd}}
\label{api:Go:GroupSetAdd}
\index{GroupSetAdd!Go API}
Add the specified value to the set for each object in \code{space} that matches
\code{checks}.

This operation will only affect objects that match the provided \code{checks}.
Objects that do not match \code{checks} will be unaffected by the group call.
Each object that matches \code{checks} will be atomically updated with the check
on the object.  HyperDex guarantees that no object will be altered if the
\code{checks} do not pass at the time of the write.  Objects that are updated
concurrently with the group call may or may not be updated; however, regardless
of any other concurrent operations, the preceding guarantee will always hold.



\paragraph{Definition:}
\begin{gocode}
func (client *Client) GroupSetAdd(spacename string, predicates []Predicate, attributes Attributes) (count uint64, err *Error)
\end{gocode}

\paragraph{Parameters:}
\begin{itemize}[noitemsep]
\item \code{spacename}\\
The name of the space as a C-string.

\item \code{predicates}\\
A set of predicates to check against.  \code{checks} is a map from the
attributes' names to the predicates to check.

\item \code{attributes}\\
The set of attributes to modify and their respective values.  \code{attrs}
points to an array of length \code{attrs\_sz}.

\end{itemize}

\paragraph{Returns:}
This function returns via the provided callback.  In the normal case, the first
argument will be the total number of objects that match the predicate.

If the operation encounters any error, the error argument will be provided and
will specify the error, in which case the first argument is undefined.


%%%%%%%%%%%%%%%%%%%% SetRemove %%%%%%%%%%%%%%%%%%%%
\pagebreak
\subsubsection{\code{SetRemove}}
\label{api:Go:SetRemove}
\index{SetRemove!Go API}
Remove the specified value from the set.  If the value is not contained within
the set, this operation will do nothing.
This operation requires a pre-existing object in order to complete successfully.
If no object exists, the operation will fail with \code{NOTFOUND}.



\paragraph{Definition:}
\begin{gocode}
func (client *Client) SetRemove(spacename string, key Value, attributes Attributes) (err *Error)
\end{gocode}

\paragraph{Parameters:}
\begin{itemize}[noitemsep]
\item \code{spacename}\\
The name of the space as a C-string.

\item \code{key}\\
The key for the operation where \code{key} is a Javascript value.

\item \code{attributes}\\
The set of attributes to modify and their respective values.  \code{attrs}
points to an array of length \code{attrs\_sz}.

\end{itemize}

\paragraph{Returns:}
This function returns via the provided callback.  In the normal case, the first
argument will indicate success or failure of the operation with one of the
following values:

\begin{itemize}[noitemsep]
\item \code{true} if the operation succeeded
\item \code{false} if any provided predicates failed
\item \code{null} if the operation requires an existing value and none exist
\end{itemize}

If the operation encounters any error, the error argument will be provided and
will specify the error, in which case the first argument is undefined.


%%%%%%%%%%%%%%%%%%%% CondSetRemove %%%%%%%%%%%%%%%%%%%%
\pagebreak
\subsubsection{\code{CondSetRemove}}
\label{api:Go:CondSetRemove}
\index{CondSetRemove!Go API}
Remove the specified value from the set if and only if the \code{checks} hold on
the object.  If the value is not contained within the set, this operation will
do nothing.
This operation requires a pre-existing object in order to complete successfully.
If no object exists, the operation will fail with \code{NOTFOUND}.


This operation will succeed if and only if the predicates specified by
\code{checks} hold on the pre-existing object.  If any of the predicates are not
true for the existing object, then the operation will have no effect and fail
with \code{CMPFAIL}.

All checks are atomic with the write.  HyperDex guarantees that no other
operation will come between validating the checks, and writing the new version
of the object..



\paragraph{Definition:}
\begin{gocode}
func (client *Client) CondSetRemove(spacename string, key Value, predicates []Predicate, attributes Attributes) (err *Error)
\end{gocode}

\paragraph{Parameters:}
\begin{itemize}[noitemsep]
\item \code{spacename}\\
The name of the space as a C-string.

\item \code{key}\\
The key for the operation where \code{key} is a Javascript value.

\item \code{predicates}\\
A set of predicates to check against.  \code{checks} is a map from the
attributes' names to the predicates to check.

\item \code{attributes}\\
The set of attributes to modify and their respective values.  \code{attrs}
points to an array of length \code{attrs\_sz}.

\end{itemize}

\paragraph{Returns:}
This function returns via the provided callback.  In the normal case, the first
argument will indicate success or failure of the operation with one of the
following values:

\begin{itemize}[noitemsep]
\item \code{true} if the operation succeeded
\item \code{false} if any provided predicates failed
\item \code{null} if the operation requires an existing value and none exist
\end{itemize}

If the operation encounters any error, the error argument will be provided and
will specify the error, in which case the first argument is undefined.


%%%%%%%%%%%%%%%%%%%% GroupSetRemove %%%%%%%%%%%%%%%%%%%%
\pagebreak
\subsubsection{\code{GroupSetRemove}}
\label{api:Go:GroupSetRemove}
\index{GroupSetRemove!Go API}
Remove the specified value from the set for each object in \code{space} that
matches \code{checks}.  If the value is not contained within the set, this
operation will do nothing.

This operation will only affect objects that match the provided \code{checks}.
Objects that do not match \code{checks} will be unaffected by the group call.
Each object that matches \code{checks} will be atomically updated with the check
on the object.  HyperDex guarantees that no object will be altered if the
\code{checks} do not pass at the time of the write.  Objects that are updated
concurrently with the group call may or may not be updated; however, regardless
of any other concurrent operations, the preceding guarantee will always hold.



\paragraph{Definition:}
\begin{gocode}
func (client *Client) GroupSetRemove(spacename string, predicates []Predicate, attributes Attributes) (count uint64, err *Error)
\end{gocode}

\paragraph{Parameters:}
\begin{itemize}[noitemsep]
\item \code{spacename}\\
The name of the space as a C-string.

\item \code{predicates}\\
A set of predicates to check against.  \code{checks} is a map from the
attributes' names to the predicates to check.

\item \code{attributes}\\
The set of attributes to modify and their respective values.  \code{attrs}
points to an array of length \code{attrs\_sz}.

\end{itemize}

\paragraph{Returns:}
This function returns via the provided callback.  In the normal case, the first
argument will be the total number of objects that match the predicate.

If the operation encounters any error, the error argument will be provided and
will specify the error, in which case the first argument is undefined.


%%%%%%%%%%%%%%%%%%%% SetIntersect %%%%%%%%%%%%%%%%%%%%
\pagebreak
\subsubsection{\code{SetIntersect}}
\label{api:Go:SetIntersect}
\index{SetIntersect!Go API}
Store the intersection of the specified set and the existing value for each
attribute.
This operation requires a pre-existing object in order to complete successfully.
If no object exists, the operation will fail with \code{NOTFOUND}.



\paragraph{Definition:}
\begin{gocode}
func (client *Client) SetIntersect(spacename string, key Value, attributes Attributes) (err *Error)
\end{gocode}

\paragraph{Parameters:}
\begin{itemize}[noitemsep]
\item \code{spacename}\\
The name of the space as a C-string.

\item \code{key}\\
The key for the operation where \code{key} is a Javascript value.

\item \code{attributes}\\
The set of attributes to modify and their respective values.  \code{attrs}
points to an array of length \code{attrs\_sz}.

\end{itemize}

\paragraph{Returns:}
This function returns via the provided callback.  In the normal case, the first
argument will indicate success or failure of the operation with one of the
following values:

\begin{itemize}[noitemsep]
\item \code{true} if the operation succeeded
\item \code{false} if any provided predicates failed
\item \code{null} if the operation requires an existing value and none exist
\end{itemize}

If the operation encounters any error, the error argument will be provided and
will specify the error, in which case the first argument is undefined.


%%%%%%%%%%%%%%%%%%%% CondSetIntersect %%%%%%%%%%%%%%%%%%%%
\pagebreak
\subsubsection{\code{CondSetIntersect}}
\label{api:Go:CondSetIntersect}
\index{CondSetIntersect!Go API}
Store the intersection of the specified set and the existing value for each
attribute if and only if the \code{checks} hold on the object.
This operation requires a pre-existing object in order to complete successfully.
If no object exists, the operation will fail with \code{NOTFOUND}.


This operation will succeed if and only if the predicates specified by
\code{checks} hold on the pre-existing object.  If any of the predicates are not
true for the existing object, then the operation will have no effect and fail
with \code{CMPFAIL}.

All checks are atomic with the write.  HyperDex guarantees that no other
operation will come between validating the checks, and writing the new version
of the object..



\paragraph{Definition:}
\begin{gocode}
func (client *Client) CondSetIntersect(spacename string, key Value, predicates []Predicate, attributes Attributes) (err *Error)
\end{gocode}

\paragraph{Parameters:}
\begin{itemize}[noitemsep]
\item \code{spacename}\\
The name of the space as a C-string.

\item \code{key}\\
The key for the operation where \code{key} is a Javascript value.

\item \code{predicates}\\
A set of predicates to check against.  \code{checks} is a map from the
attributes' names to the predicates to check.

\item \code{attributes}\\
The set of attributes to modify and their respective values.  \code{attrs}
points to an array of length \code{attrs\_sz}.

\end{itemize}

\paragraph{Returns:}
This function returns via the provided callback.  In the normal case, the first
argument will indicate success or failure of the operation with one of the
following values:

\begin{itemize}[noitemsep]
\item \code{true} if the operation succeeded
\item \code{false} if any provided predicates failed
\item \code{null} if the operation requires an existing value and none exist
\end{itemize}

If the operation encounters any error, the error argument will be provided and
will specify the error, in which case the first argument is undefined.


%%%%%%%%%%%%%%%%%%%% GroupSetIntersect %%%%%%%%%%%%%%%%%%%%
\pagebreak
\subsubsection{\code{GroupSetIntersect}}
\label{api:Go:GroupSetIntersect}
\index{GroupSetIntersect!Go API}
Store the intersection of the specified set and the existing value for each
object in \code{space} that matches \code{checks}.

This operation will only affect objects that match the provided \code{checks}.
Objects that do not match \code{checks} will be unaffected by the group call.
Each object that matches \code{checks} will be atomically updated with the check
on the object.  HyperDex guarantees that no object will be altered if the
\code{checks} do not pass at the time of the write.  Objects that are updated
concurrently with the group call may or may not be updated; however, regardless
of any other concurrent operations, the preceding guarantee will always hold.



\paragraph{Definition:}
\begin{gocode}
func (client *Client) GroupSetIntersect(spacename string, predicates []Predicate, attributes Attributes) (count uint64, err *Error)
\end{gocode}

\paragraph{Parameters:}
\begin{itemize}[noitemsep]
\item \code{spacename}\\
The name of the space as a C-string.

\item \code{predicates}\\
A set of predicates to check against.  \code{checks} is a map from the
attributes' names to the predicates to check.

\item \code{attributes}\\
The set of attributes to modify and their respective values.  \code{attrs}
points to an array of length \code{attrs\_sz}.

\end{itemize}

\paragraph{Returns:}
This function returns via the provided callback.  In the normal case, the first
argument will be the total number of objects that match the predicate.

If the operation encounters any error, the error argument will be provided and
will specify the error, in which case the first argument is undefined.


%%%%%%%%%%%%%%%%%%%% SetUnion %%%%%%%%%%%%%%%%%%%%
\pagebreak
\subsubsection{\code{SetUnion}}
\label{api:Go:SetUnion}
\index{SetUnion!Go API}
Store the union of the specified set and the existing value for each attribute.
This operation requires a pre-existing object in order to complete successfully.
If no object exists, the operation will fail with \code{NOTFOUND}.



\paragraph{Definition:}
\begin{gocode}
func (client *Client) SetUnion(spacename string, key Value, attributes Attributes) (err *Error)
\end{gocode}

\paragraph{Parameters:}
\begin{itemize}[noitemsep]
\item \code{spacename}\\
The name of the space as a C-string.

\item \code{key}\\
The key for the operation where \code{key} is a Javascript value.

\item \code{attributes}\\
The set of attributes to modify and their respective values.  \code{attrs}
points to an array of length \code{attrs\_sz}.

\end{itemize}

\paragraph{Returns:}
This function returns via the provided callback.  In the normal case, the first
argument will indicate success or failure of the operation with one of the
following values:

\begin{itemize}[noitemsep]
\item \code{true} if the operation succeeded
\item \code{false} if any provided predicates failed
\item \code{null} if the operation requires an existing value and none exist
\end{itemize}

If the operation encounters any error, the error argument will be provided and
will specify the error, in which case the first argument is undefined.


%%%%%%%%%%%%%%%%%%%% CondSetUnion %%%%%%%%%%%%%%%%%%%%
\pagebreak
\subsubsection{\code{CondSetUnion}}
\label{api:Go:CondSetUnion}
\index{CondSetUnion!Go API}
Store the union of the specified set and the existing value for each attribute
if and only if the \code{checks} hold on the object.
This operation requires a pre-existing object in order to complete successfully.
If no object exists, the operation will fail with \code{NOTFOUND}.


This operation will succeed if and only if the predicates specified by
\code{checks} hold on the pre-existing object.  If any of the predicates are not
true for the existing object, then the operation will have no effect and fail
with \code{CMPFAIL}.

All checks are atomic with the write.  HyperDex guarantees that no other
operation will come between validating the checks, and writing the new version
of the object..



\paragraph{Definition:}
\begin{gocode}
func (client *Client) CondSetUnion(spacename string, key Value, predicates []Predicate, attributes Attributes) (err *Error)
\end{gocode}

\paragraph{Parameters:}
\begin{itemize}[noitemsep]
\item \code{spacename}\\
The name of the space as a C-string.

\item \code{key}\\
The key for the operation where \code{key} is a Javascript value.

\item \code{predicates}\\
A set of predicates to check against.  \code{checks} is a map from the
attributes' names to the predicates to check.

\item \code{attributes}\\
The set of attributes to modify and their respective values.  \code{attrs}
points to an array of length \code{attrs\_sz}.

\end{itemize}

\paragraph{Returns:}
This function returns via the provided callback.  In the normal case, the first
argument will indicate success or failure of the operation with one of the
following values:

\begin{itemize}[noitemsep]
\item \code{true} if the operation succeeded
\item \code{false} if any provided predicates failed
\item \code{null} if the operation requires an existing value and none exist
\end{itemize}

If the operation encounters any error, the error argument will be provided and
will specify the error, in which case the first argument is undefined.


%%%%%%%%%%%%%%%%%%%% GroupSetUnion %%%%%%%%%%%%%%%%%%%%
\pagebreak
\subsubsection{\code{GroupSetUnion}}
\label{api:Go:GroupSetUnion}
\index{GroupSetUnion!Go API}
Store the union of the specified set and the existing value for each object in
\code{space} that matches \code{checks}.

This operation will only affect objects that match the provided \code{checks}.
Objects that do not match \code{checks} will be unaffected by the group call.
Each object that matches \code{checks} will be atomically updated with the check
on the object.  HyperDex guarantees that no object will be altered if the
\code{checks} do not pass at the time of the write.  Objects that are updated
concurrently with the group call may or may not be updated; however, regardless
of any other concurrent operations, the preceding guarantee will always hold.



\paragraph{Definition:}
\begin{gocode}
func (client *Client) GroupSetUnion(spacename string, predicates []Predicate, attributes Attributes) (count uint64, err *Error)
\end{gocode}

\paragraph{Parameters:}
\begin{itemize}[noitemsep]
\item \code{spacename}\\
The name of the space as a C-string.

\item \code{predicates}\\
A set of predicates to check against.  \code{checks} is a map from the
attributes' names to the predicates to check.

\item \code{attributes}\\
The set of attributes to modify and their respective values.  \code{attrs}
points to an array of length \code{attrs\_sz}.

\end{itemize}

\paragraph{Returns:}
This function returns via the provided callback.  In the normal case, the first
argument will be the total number of objects that match the predicate.

If the operation encounters any error, the error argument will be provided and
will specify the error, in which case the first argument is undefined.


%%%%%%%%%%%%%%%%%%%% DocumentRename %%%%%%%%%%%%%%%%%%%%
\pagebreak
\subsubsection{\code{DocumentRename}}
\label{api:Go:DocumentRename}
\index{DocumentRename!Go API}
Move a field within a document from one name to another.
This operation requires a pre-existing object in order to complete successfully.
If no object exists, the operation will fail with \code{NOTFOUND}.



\paragraph{Definition:}
\begin{gocode}
func (client *Client) DocumentRename(spacename string, key Value, attributes Attributes) (err *Error)
\end{gocode}

\paragraph{Parameters:}
\begin{itemize}[noitemsep]
\item \code{spacename}\\
The name of the space as a C-string.

\item \code{key}\\
The key for the operation where \code{key} is a Javascript value.

\item \code{attributes}\\
The set of attributes to modify and their respective values.  \code{attrs}
points to an array of length \code{attrs\_sz}.

\end{itemize}

\paragraph{Returns:}
This function returns via the provided callback.  In the normal case, the first
argument will indicate success or failure of the operation with one of the
following values:

\begin{itemize}[noitemsep]
\item \code{true} if the operation succeeded
\item \code{false} if any provided predicates failed
\item \code{null} if the operation requires an existing value and none exist
\end{itemize}

If the operation encounters any error, the error argument will be provided and
will specify the error, in which case the first argument is undefined.


%%%%%%%%%%%%%%%%%%%% CondDocumentRename %%%%%%%%%%%%%%%%%%%%
\pagebreak
\subsubsection{\code{CondDocumentRename}}
\label{api:Go:CondDocumentRename}
\index{CondDocumentRename!Go API}
Move a field within a document from one name to another if and only if the
\code{checks} hold on the object.
This operation requires a pre-existing object in order to complete successfully.
If no object exists, the operation will fail with \code{NOTFOUND}.


This operation will succeed if and only if the predicates specified by
\code{checks} hold on the pre-existing object.  If any of the predicates are not
true for the existing object, then the operation will have no effect and fail
with \code{CMPFAIL}.

All checks are atomic with the write.  HyperDex guarantees that no other
operation will come between validating the checks, and writing the new version
of the object..



\paragraph{Definition:}
\begin{gocode}
func (client *Client) CondDocumentRename(spacename string, key Value, predicates []Predicate, attributes Attributes) (err *Error)
\end{gocode}

\paragraph{Parameters:}
\begin{itemize}[noitemsep]
\item \code{spacename}\\
The name of the space as a C-string.

\item \code{key}\\
The key for the operation where \code{key} is a Javascript value.

\item \code{predicates}\\
A set of predicates to check against.  \code{checks} is a map from the
attributes' names to the predicates to check.

\item \code{attributes}\\
The set of attributes to modify and their respective values.  \code{attrs}
points to an array of length \code{attrs\_sz}.

\end{itemize}

\paragraph{Returns:}
This function returns via the provided callback.  In the normal case, the first
argument will indicate success or failure of the operation with one of the
following values:

\begin{itemize}[noitemsep]
\item \code{true} if the operation succeeded
\item \code{false} if any provided predicates failed
\item \code{null} if the operation requires an existing value and none exist
\end{itemize}

If the operation encounters any error, the error argument will be provided and
will specify the error, in which case the first argument is undefined.


%%%%%%%%%%%%%%%%%%%% GroupDocumentRename %%%%%%%%%%%%%%%%%%%%
\pagebreak
\subsubsection{\code{GroupDocumentRename}}
\label{api:Go:GroupDocumentRename}
\index{GroupDocumentRename!Go API}
Move a field within a document from one name to another for each object in
\code{space} that matches \code{checks}.

This operation will only affect objects that match the provided \code{checks}.
Objects that do not match \code{checks} will be unaffected by the group call.
Each object that matches \code{checks} will be atomically updated with the check
on the object.  HyperDex guarantees that no object will be altered if the
\code{checks} do not pass at the time of the write.  Objects that are updated
concurrently with the group call may or may not be updated; however, regardless
of any other concurrent operations, the preceding guarantee will always hold.



\paragraph{Definition:}
\begin{gocode}
func (client *Client) GroupDocumentRename(spacename string, predicates []Predicate, attributes Attributes) (count uint64, err *Error)
\end{gocode}

\paragraph{Parameters:}
\begin{itemize}[noitemsep]
\item \code{spacename}\\
The name of the space as a C-string.

\item \code{predicates}\\
A set of predicates to check against.  \code{checks} is a map from the
attributes' names to the predicates to check.

\item \code{attributes}\\
The set of attributes to modify and their respective values.  \code{attrs}
points to an array of length \code{attrs\_sz}.

\end{itemize}

\paragraph{Returns:}
This function returns via the provided callback.  In the normal case, the first
argument will be the total number of objects that match the predicate.

If the operation encounters any error, the error argument will be provided and
will specify the error, in which case the first argument is undefined.


%%%%%%%%%%%%%%%%%%%% DocumentUnset %%%%%%%%%%%%%%%%%%%%
\pagebreak
\subsubsection{\code{DocumentUnset}}
\label{api:Go:DocumentUnset}
\index{DocumentUnset!Go API}
Remove a field or object from a document.
This operation requires a pre-existing object in order to complete successfully.
If no object exists, the operation will fail with \code{NOTFOUND}.



\paragraph{Definition:}
\begin{gocode}
func (client *Client) DocumentUnset(spacename string, key Value, attributes Attributes) (err *Error)
\end{gocode}

\paragraph{Parameters:}
\begin{itemize}[noitemsep]
\item \code{spacename}\\
The name of the space as a C-string.

\item \code{key}\\
The key for the operation where \code{key} is a Javascript value.

\item \code{attributes}\\
The set of attributes to modify and their respective values.  \code{attrs}
points to an array of length \code{attrs\_sz}.

\end{itemize}

\paragraph{Returns:}
This function returns via the provided callback.  In the normal case, the first
argument will indicate success or failure of the operation with one of the
following values:

\begin{itemize}[noitemsep]
\item \code{true} if the operation succeeded
\item \code{false} if any provided predicates failed
\item \code{null} if the operation requires an existing value and none exist
\end{itemize}

If the operation encounters any error, the error argument will be provided and
will specify the error, in which case the first argument is undefined.


%%%%%%%%%%%%%%%%%%%% CondDocumentUnset %%%%%%%%%%%%%%%%%%%%
\pagebreak
\subsubsection{\code{CondDocumentUnset}}
\label{api:Go:CondDocumentUnset}
\index{CondDocumentUnset!Go API}
Remove a field or object from a document if and only if the \code{checks} hold
on the object.
This operation requires a pre-existing object in order to complete successfully.
If no object exists, the operation will fail with \code{NOTFOUND}.


This operation will succeed if and only if the predicates specified by
\code{checks} hold on the pre-existing object.  If any of the predicates are not
true for the existing object, then the operation will have no effect and fail
with \code{CMPFAIL}.

All checks are atomic with the write.  HyperDex guarantees that no other
operation will come between validating the checks, and writing the new version
of the object..



\paragraph{Definition:}
\begin{gocode}
func (client *Client) CondDocumentUnset(spacename string, key Value, predicates []Predicate, attributes Attributes) (err *Error)
\end{gocode}

\paragraph{Parameters:}
\begin{itemize}[noitemsep]
\item \code{spacename}\\
The name of the space as a C-string.

\item \code{key}\\
The key for the operation where \code{key} is a Javascript value.

\item \code{predicates}\\
A set of predicates to check against.  \code{checks} is a map from the
attributes' names to the predicates to check.

\item \code{attributes}\\
The set of attributes to modify and their respective values.  \code{attrs}
points to an array of length \code{attrs\_sz}.

\end{itemize}

\paragraph{Returns:}
This function returns via the provided callback.  In the normal case, the first
argument will indicate success or failure of the operation with one of the
following values:

\begin{itemize}[noitemsep]
\item \code{true} if the operation succeeded
\item \code{false} if any provided predicates failed
\item \code{null} if the operation requires an existing value and none exist
\end{itemize}

If the operation encounters any error, the error argument will be provided and
will specify the error, in which case the first argument is undefined.


%%%%%%%%%%%%%%%%%%%% GroupDocumentUnset %%%%%%%%%%%%%%%%%%%%
\pagebreak
\subsubsection{\code{GroupDocumentUnset}}
\label{api:Go:GroupDocumentUnset}
\index{GroupDocumentUnset!Go API}
Remove a field or object from a document for each object in \code{space} that
matches \code{checks}.

This operation will only affect objects that match the provided \code{checks}.
Objects that do not match \code{checks} will be unaffected by the group call.
Each object that matches \code{checks} will be atomically updated with the check
on the object.  HyperDex guarantees that no object will be altered if the
\code{checks} do not pass at the time of the write.  Objects that are updated
concurrently with the group call may or may not be updated; however, regardless
of any other concurrent operations, the preceding guarantee will always hold.



\paragraph{Definition:}
\begin{gocode}
func (client *Client) GroupDocumentUnset(spacename string, predicates []Predicate, attributes Attributes) (count uint64, err *Error)
\end{gocode}

\paragraph{Parameters:}
\begin{itemize}[noitemsep]
\item \code{spacename}\\
The name of the space as a C-string.

\item \code{predicates}\\
A set of predicates to check against.  \code{checks} is a map from the
attributes' names to the predicates to check.

\item \code{attributes}\\
The set of attributes to modify and their respective values.  \code{attrs}
points to an array of length \code{attrs\_sz}.

\end{itemize}

\paragraph{Returns:}
This function returns via the provided callback.  In the normal case, the first
argument will be the total number of objects that match the predicate.

If the operation encounters any error, the error argument will be provided and
will specify the error, in which case the first argument is undefined.


%%%%%%%%%%%%%%%%%%%% MapAdd %%%%%%%%%%%%%%%%%%%%
\pagebreak
\subsubsection{\code{MapAdd}}
\label{api:Go:MapAdd}
\index{MapAdd!Go API}
Insert a key-value pair into the map specified by each map-attribute.
This operation requires a pre-existing object in order to complete successfully.
If no object exists, the operation will fail with \code{NOTFOUND}.



\paragraph{Definition:}
\begin{gocode}
func (client *Client) MapAdd(spacename string, key Value, mapattributes MapAttributes) (err *Error)
\end{gocode}

\paragraph{Parameters:}
\begin{itemize}[noitemsep]
\item \code{spacename}\\
The name of the space as a C-string.

\item \code{key}\\
The key for the operation where \code{key} is a Javascript value.

\item \code{mapattributes}\\
The set of attributes to modify and their respective values.  \code{mapattrs} is
a map from the attributes' names to inner maps which contain the key-value pairs
to be modified

\end{itemize}

\paragraph{Returns:}
This function returns via the provided callback.  In the normal case, the first
argument will indicate success or failure of the operation with one of the
following values:

\begin{itemize}[noitemsep]
\item \code{true} if the operation succeeded
\item \code{false} if any provided predicates failed
\item \code{null} if the operation requires an existing value and none exist
\end{itemize}

If the operation encounters any error, the error argument will be provided and
will specify the error, in which case the first argument is undefined.


%%%%%%%%%%%%%%%%%%%% CondMapAdd %%%%%%%%%%%%%%%%%%%%
\pagebreak
\subsubsection{\code{CondMapAdd}}
\label{api:Go:CondMapAdd}
\index{CondMapAdd!Go API}
Insert a key-value pair into the map specified by each map-attribute if and only
if the \code{checks} hold on the object.
This operation requires a pre-existing object in order to complete successfully.
If no object exists, the operation will fail with \code{NOTFOUND}.


This operation will succeed if and only if the predicates specified by
\code{checks} hold on the pre-existing object.  If any of the predicates are not
true for the existing object, then the operation will have no effect and fail
with \code{CMPFAIL}.

All checks are atomic with the write.  HyperDex guarantees that no other
operation will come between validating the checks, and writing the new version
of the object..



\paragraph{Definition:}
\begin{gocode}
func (client *Client) CondMapAdd(spacename string, key Value, predicates []Predicate, mapattributes MapAttributes) (err *Error)
\end{gocode}

\paragraph{Parameters:}
\begin{itemize}[noitemsep]
\item \code{spacename}\\
The name of the space as a C-string.

\item \code{key}\\
The key for the operation where \code{key} is a Javascript value.

\item \code{predicates}\\
A set of predicates to check against.  \code{checks} is a map from the
attributes' names to the predicates to check.

\item \code{mapattributes}\\
The set of attributes to modify and their respective values.  \code{mapattrs} is
a map from the attributes' names to inner maps which contain the key-value pairs
to be modified

\end{itemize}

\paragraph{Returns:}
This function returns via the provided callback.  In the normal case, the first
argument will indicate success or failure of the operation with one of the
following values:

\begin{itemize}[noitemsep]
\item \code{true} if the operation succeeded
\item \code{false} if any provided predicates failed
\item \code{null} if the operation requires an existing value and none exist
\end{itemize}

If the operation encounters any error, the error argument will be provided and
will specify the error, in which case the first argument is undefined.


%%%%%%%%%%%%%%%%%%%% GroupMapAdd %%%%%%%%%%%%%%%%%%%%
\pagebreak
\subsubsection{\code{GroupMapAdd}}
\label{api:Go:GroupMapAdd}
\index{GroupMapAdd!Go API}
Insert a key-value pair into the map specified by each map-attribute for each
object in \code{space} that matches \code{checks}.

This operation will only affect objects that match the provided \code{checks}.
Objects that do not match \code{checks} will be unaffected by the group call.
Each object that matches \code{checks} will be atomically updated with the check
on the object.  HyperDex guarantees that no object will be altered if the
\code{checks} do not pass at the time of the write.  Objects that are updated
concurrently with the group call may or may not be updated; however, regardless
of any other concurrent operations, the preceding guarantee will always hold.



\paragraph{Definition:}
\begin{gocode}
func (client *Client) GroupMapAdd(spacename string, predicates []Predicate, mapattributes MapAttributes) (count uint64, err *Error)
\end{gocode}

\paragraph{Parameters:}
\begin{itemize}[noitemsep]
\item \code{spacename}\\
The name of the space as a C-string.

\item \code{predicates}\\
A set of predicates to check against.  \code{checks} is a map from the
attributes' names to the predicates to check.

\item \code{mapattributes}\\
The set of attributes to modify and their respective values.  \code{mapattrs} is
a map from the attributes' names to inner maps which contain the key-value pairs
to be modified

\end{itemize}

\paragraph{Returns:}
This function returns via the provided callback.  In the normal case, the first
argument will be the total number of objects that match the predicate.

If the operation encounters any error, the error argument will be provided and
will specify the error, in which case the first argument is undefined.


%%%%%%%%%%%%%%%%%%%% MapRemove %%%%%%%%%%%%%%%%%%%%
\pagebreak
\subsubsection{\code{MapRemove}}
\label{api:Go:MapRemove}
\index{MapRemove!Go API}
Remove a key-value pair from the map specified by each attribute.  If there is
no pair with the specified key within the map, this operation will do nothing.
This operation requires a pre-existing object in order to complete successfully.
If no object exists, the operation will fail with \code{NOTFOUND}.



\paragraph{Definition:}
\begin{gocode}
func (client *Client) MapRemove(spacename string, key Value, attributes Attributes) (err *Error)
\end{gocode}

\paragraph{Parameters:}
\begin{itemize}[noitemsep]
\item \code{spacename}\\
The name of the space as a C-string.

\item \code{key}\\
The key for the operation where \code{key} is a Javascript value.

\item \code{attributes}\\
The set of attributes to modify and their respective values.  \code{attrs}
points to an array of length \code{attrs\_sz}.

\end{itemize}

\paragraph{Returns:}
This function returns via the provided callback.  In the normal case, the first
argument will indicate success or failure of the operation with one of the
following values:

\begin{itemize}[noitemsep]
\item \code{true} if the operation succeeded
\item \code{false} if any provided predicates failed
\item \code{null} if the operation requires an existing value and none exist
\end{itemize}

If the operation encounters any error, the error argument will be provided and
will specify the error, in which case the first argument is undefined.


%%%%%%%%%%%%%%%%%%%% CondMapRemove %%%%%%%%%%%%%%%%%%%%
\pagebreak
\subsubsection{\code{CondMapRemove}}
\label{api:Go:CondMapRemove}
\index{CondMapRemove!Go API}
Remove a key-value pair from the map specified by each attribute if and only if
\code{checks} hold on the object.  If there is no pair with the specified key
within the map, this operation will do nothing.
This operation requires a pre-existing object in order to complete successfully.
If no object exists, the operation will fail with \code{NOTFOUND}.


This operation will succeed if and only if the predicates specified by
\code{checks} hold on the pre-existing object.  If any of the predicates are not
true for the existing object, then the operation will have no effect and fail
with \code{CMPFAIL}.

All checks are atomic with the write.  HyperDex guarantees that no other
operation will come between validating the checks, and writing the new version
of the object..



\paragraph{Definition:}
\begin{gocode}
func (client *Client) CondMapRemove(spacename string, key Value, predicates []Predicate, attributes Attributes) (err *Error)
\end{gocode}

\paragraph{Parameters:}
\begin{itemize}[noitemsep]
\item \code{spacename}\\
The name of the space as a C-string.

\item \code{key}\\
The key for the operation where \code{key} is a Javascript value.

\item \code{predicates}\\
A set of predicates to check against.  \code{checks} is a map from the
attributes' names to the predicates to check.

\item \code{attributes}\\
The set of attributes to modify and their respective values.  \code{attrs}
points to an array of length \code{attrs\_sz}.

\end{itemize}

\paragraph{Returns:}
This function returns via the provided callback.  In the normal case, the first
argument will indicate success or failure of the operation with one of the
following values:

\begin{itemize}[noitemsep]
\item \code{true} if the operation succeeded
\item \code{false} if any provided predicates failed
\item \code{null} if the operation requires an existing value and none exist
\end{itemize}

If the operation encounters any error, the error argument will be provided and
will specify the error, in which case the first argument is undefined.


%%%%%%%%%%%%%%%%%%%% GroupMapRemove %%%%%%%%%%%%%%%%%%%%
\pagebreak
\subsubsection{\code{GroupMapRemove}}
\label{api:Go:GroupMapRemove}
\index{GroupMapRemove!Go API}
Remove a key-value pair from the map specified by each attribute for each object
in \code{space} that matches \code{checks}.  If there is no pair with the
specified key within the map, this operation will do nothing.

This operation will only affect objects that match the provided \code{checks}.
Objects that do not match \code{checks} will be unaffected by the group call.
Each object that matches \code{checks} will be atomically updated with the check
on the object.  HyperDex guarantees that no object will be altered if the
\code{checks} do not pass at the time of the write.  Objects that are updated
concurrently with the group call may or may not be updated; however, regardless
of any other concurrent operations, the preceding guarantee will always hold.



\paragraph{Definition:}
\begin{gocode}
func (client *Client) GroupMapRemove(spacename string, predicates []Predicate, attributes Attributes) (count uint64, err *Error)
\end{gocode}

\paragraph{Parameters:}
\begin{itemize}[noitemsep]
\item \code{spacename}\\
The name of the space as a C-string.

\item \code{predicates}\\
A set of predicates to check against.  \code{checks} is a map from the
attributes' names to the predicates to check.

\item \code{attributes}\\
The set of attributes to modify and their respective values.  \code{attrs}
points to an array of length \code{attrs\_sz}.

\end{itemize}

\paragraph{Returns:}
This function returns via the provided callback.  In the normal case, the first
argument will be the total number of objects that match the predicate.

If the operation encounters any error, the error argument will be provided and
will specify the error, in which case the first argument is undefined.


%%%%%%%%%%%%%%%%%%%% MapAtomicAdd %%%%%%%%%%%%%%%%%%%%
\pagebreak
\subsubsection{\code{MapAtomicAdd}}
\label{api:Go:MapAtomicAdd}
\index{MapAtomicAdd!Go API}
Add the specified number to the value of a key-value pair within each map.
This operation requires a pre-existing object in order to complete successfully.
If no object exists, the operation will fail with \code{NOTFOUND}.



\paragraph{Definition:}
\begin{gocode}
func (client *Client) MapAtomicAdd(spacename string, key Value, mapattributes MapAttributes) (err *Error)
\end{gocode}

\paragraph{Parameters:}
\begin{itemize}[noitemsep]
\item \code{spacename}\\
The name of the space as a C-string.

\item \code{key}\\
The key for the operation where \code{key} is a Javascript value.

\item \code{mapattributes}\\
The set of attributes to modify and their respective values.  \code{mapattrs} is
a map from the attributes' names to inner maps which contain the key-value pairs
to be modified

\end{itemize}

\paragraph{Returns:}
This function returns via the provided callback.  In the normal case, the first
argument will indicate success or failure of the operation with one of the
following values:

\begin{itemize}[noitemsep]
\item \code{true} if the operation succeeded
\item \code{false} if any provided predicates failed
\item \code{null} if the operation requires an existing value and none exist
\end{itemize}

If the operation encounters any error, the error argument will be provided and
will specify the error, in which case the first argument is undefined.


%%%%%%%%%%%%%%%%%%%% CondMapAtomicAdd %%%%%%%%%%%%%%%%%%%%
\pagebreak
\subsubsection{\code{CondMapAtomicAdd}}
\label{api:Go:CondMapAtomicAdd}
\index{CondMapAtomicAdd!Go API}
Add the specified number to the value of a key-value pair within each map if and
only if the \code{checks} hold on the object.
This operation requires a pre-existing object in order to complete successfully.
If no object exists, the operation will fail with \code{NOTFOUND}.


This operation will succeed if and only if the predicates specified by
\code{checks} hold on the pre-existing object.  If any of the predicates are not
true for the existing object, then the operation will have no effect and fail
with \code{CMPFAIL}.

All checks are atomic with the write.  HyperDex guarantees that no other
operation will come between validating the checks, and writing the new version
of the object..



\paragraph{Definition:}
\begin{gocode}
func (client *Client) CondMapAtomicAdd(spacename string, key Value, predicates []Predicate, mapattributes MapAttributes) (err *Error)
\end{gocode}

\paragraph{Parameters:}
\begin{itemize}[noitemsep]
\item \code{spacename}\\
The name of the space as a C-string.

\item \code{key}\\
The key for the operation where \code{key} is a Javascript value.

\item \code{predicates}\\
A set of predicates to check against.  \code{checks} is a map from the
attributes' names to the predicates to check.

\item \code{mapattributes}\\
The set of attributes to modify and their respective values.  \code{mapattrs} is
a map from the attributes' names to inner maps which contain the key-value pairs
to be modified

\end{itemize}

\paragraph{Returns:}
This function returns via the provided callback.  In the normal case, the first
argument will indicate success or failure of the operation with one of the
following values:

\begin{itemize}[noitemsep]
\item \code{true} if the operation succeeded
\item \code{false} if any provided predicates failed
\item \code{null} if the operation requires an existing value and none exist
\end{itemize}

If the operation encounters any error, the error argument will be provided and
will specify the error, in which case the first argument is undefined.


%%%%%%%%%%%%%%%%%%%% GroupMapAtomicAdd %%%%%%%%%%%%%%%%%%%%
\pagebreak
\subsubsection{\code{GroupMapAtomicAdd}}
\label{api:Go:GroupMapAtomicAdd}
\index{GroupMapAtomicAdd!Go API}
Add the specified number to the value of a key-value pair within each map for
each object in \code{space} that matches \code{checks}.

This operation will only affect objects that match the provided \code{checks}.
Objects that do not match \code{checks} will be unaffected by the group call.
Each object that matches \code{checks} will be atomically updated with the check
on the object.  HyperDex guarantees that no object will be altered if the
\code{checks} do not pass at the time of the write.  Objects that are updated
concurrently with the group call may or may not be updated; however, regardless
of any other concurrent operations, the preceding guarantee will always hold.



\paragraph{Definition:}
\begin{gocode}
func (client *Client) GroupMapAtomicAdd(spacename string, predicates []Predicate, mapattributes MapAttributes) (count uint64, err *Error)
\end{gocode}

\paragraph{Parameters:}
\begin{itemize}[noitemsep]
\item \code{spacename}\\
The name of the space as a C-string.

\item \code{predicates}\\
A set of predicates to check against.  \code{checks} is a map from the
attributes' names to the predicates to check.

\item \code{mapattributes}\\
The set of attributes to modify and their respective values.  \code{mapattrs} is
a map from the attributes' names to inner maps which contain the key-value pairs
to be modified

\end{itemize}

\paragraph{Returns:}
This function returns via the provided callback.  In the normal case, the first
argument will be the total number of objects that match the predicate.

If the operation encounters any error, the error argument will be provided and
will specify the error, in which case the first argument is undefined.


%%%%%%%%%%%%%%%%%%%% MapAtomicSub %%%%%%%%%%%%%%%%%%%%
\pagebreak
\subsubsection{\code{MapAtomicSub}}
\label{api:Go:MapAtomicSub}
\index{MapAtomicSub!Go API}
Subtract the specified number from the value of a key-value pair within each
map.
This operation requires a pre-existing object in order to complete successfully.
If no object exists, the operation will fail with \code{NOTFOUND}.



\paragraph{Definition:}
\begin{gocode}
func (client *Client) MapAtomicSub(spacename string, key Value, mapattributes MapAttributes) (err *Error)
\end{gocode}

\paragraph{Parameters:}
\begin{itemize}[noitemsep]
\item \code{spacename}\\
The name of the space as a C-string.

\item \code{key}\\
The key for the operation where \code{key} is a Javascript value.

\item \code{mapattributes}\\
The set of attributes to modify and their respective values.  \code{mapattrs} is
a map from the attributes' names to inner maps which contain the key-value pairs
to be modified

\end{itemize}

\paragraph{Returns:}
This function returns via the provided callback.  In the normal case, the first
argument will indicate success or failure of the operation with one of the
following values:

\begin{itemize}[noitemsep]
\item \code{true} if the operation succeeded
\item \code{false} if any provided predicates failed
\item \code{null} if the operation requires an existing value and none exist
\end{itemize}

If the operation encounters any error, the error argument will be provided and
will specify the error, in which case the first argument is undefined.


%%%%%%%%%%%%%%%%%%%% CondMapAtomicSub %%%%%%%%%%%%%%%%%%%%
\pagebreak
\subsubsection{\code{CondMapAtomicSub}}
\label{api:Go:CondMapAtomicSub}
\index{CondMapAtomicSub!Go API}
Subtract the specified number from the value of a key-value pair within each
map if and only if the \code{checks} hold on the object.
This operation requires a pre-existing object in order to complete successfully.
If no object exists, the operation will fail with \code{NOTFOUND}.


This operation will succeed if and only if the predicates specified by
\code{checks} hold on the pre-existing object.  If any of the predicates are not
true for the existing object, then the operation will have no effect and fail
with \code{CMPFAIL}.

All checks are atomic with the write.  HyperDex guarantees that no other
operation will come between validating the checks, and writing the new version
of the object..



\paragraph{Definition:}
\begin{gocode}
func (client *Client) CondMapAtomicSub(spacename string, key Value, predicates []Predicate, mapattributes MapAttributes) (err *Error)
\end{gocode}

\paragraph{Parameters:}
\begin{itemize}[noitemsep]
\item \code{spacename}\\
The name of the space as a C-string.

\item \code{key}\\
The key for the operation where \code{key} is a Javascript value.

\item \code{predicates}\\
A set of predicates to check against.  \code{checks} is a map from the
attributes' names to the predicates to check.

\item \code{mapattributes}\\
The set of attributes to modify and their respective values.  \code{mapattrs} is
a map from the attributes' names to inner maps which contain the key-value pairs
to be modified

\end{itemize}

\paragraph{Returns:}
This function returns via the provided callback.  In the normal case, the first
argument will indicate success or failure of the operation with one of the
following values:

\begin{itemize}[noitemsep]
\item \code{true} if the operation succeeded
\item \code{false} if any provided predicates failed
\item \code{null} if the operation requires an existing value and none exist
\end{itemize}

If the operation encounters any error, the error argument will be provided and
will specify the error, in which case the first argument is undefined.


%%%%%%%%%%%%%%%%%%%% GroupMapAtomicSub %%%%%%%%%%%%%%%%%%%%
\pagebreak
\subsubsection{\code{GroupMapAtomicSub}}
\label{api:Go:GroupMapAtomicSub}
\index{GroupMapAtomicSub!Go API}
Subtract the specified number from the value of a key-value pair within each
map for each object in \code{space} that matches \code{checks}.

This operation will only affect objects that match the provided \code{checks}.
Objects that do not match \code{checks} will be unaffected by the group call.
Each object that matches \code{checks} will be atomically updated with the check
on the object.  HyperDex guarantees that no object will be altered if the
\code{checks} do not pass at the time of the write.  Objects that are updated
concurrently with the group call may or may not be updated; however, regardless
of any other concurrent operations, the preceding guarantee will always hold.



\paragraph{Definition:}
\begin{gocode}
func (client *Client) GroupMapAtomicSub(spacename string, predicates []Predicate, mapattributes MapAttributes) (count uint64, err *Error)
\end{gocode}

\paragraph{Parameters:}
\begin{itemize}[noitemsep]
\item \code{spacename}\\
The name of the space as a C-string.

\item \code{predicates}\\
A set of predicates to check against.  \code{checks} is a map from the
attributes' names to the predicates to check.

\item \code{mapattributes}\\
The set of attributes to modify and their respective values.  \code{mapattrs} is
a map from the attributes' names to inner maps which contain the key-value pairs
to be modified

\end{itemize}

\paragraph{Returns:}
This function returns via the provided callback.  In the normal case, the first
argument will be the total number of objects that match the predicate.

If the operation encounters any error, the error argument will be provided and
will specify the error, in which case the first argument is undefined.


%%%%%%%%%%%%%%%%%%%% MapAtomicMul %%%%%%%%%%%%%%%%%%%%
\pagebreak
\subsubsection{\code{MapAtomicMul}}
\label{api:Go:MapAtomicMul}
\index{MapAtomicMul!Go API}
Multiply the value of each key-value pair by the specified number for each map.
This operation requires a pre-existing object in order to complete successfully.
If no object exists, the operation will fail with \code{NOTFOUND}.



\paragraph{Definition:}
\begin{gocode}
func (client *Client) MapAtomicMul(spacename string, key Value, mapattributes MapAttributes) (err *Error)
\end{gocode}

\paragraph{Parameters:}
\begin{itemize}[noitemsep]
\item \code{spacename}\\
The name of the space as a C-string.

\item \code{key}\\
The key for the operation where \code{key} is a Javascript value.

\item \code{mapattributes}\\
The set of attributes to modify and their respective values.  \code{mapattrs} is
a map from the attributes' names to inner maps which contain the key-value pairs
to be modified

\end{itemize}

\paragraph{Returns:}
This function returns via the provided callback.  In the normal case, the first
argument will indicate success or failure of the operation with one of the
following values:

\begin{itemize}[noitemsep]
\item \code{true} if the operation succeeded
\item \code{false} if any provided predicates failed
\item \code{null} if the operation requires an existing value and none exist
\end{itemize}

If the operation encounters any error, the error argument will be provided and
will specify the error, in which case the first argument is undefined.


%%%%%%%%%%%%%%%%%%%% CondMapAtomicMul %%%%%%%%%%%%%%%%%%%%
\pagebreak
\subsubsection{\code{CondMapAtomicMul}}
\label{api:Go:CondMapAtomicMul}
\index{CondMapAtomicMul!Go API}
Multiply the value of each key-value pair by the specified number for each map
attribute if and only if the \code{checks} hold on the object.
This operation requires a pre-existing object in order to complete successfully.
If no object exists, the operation will fail with \code{NOTFOUND}.


This operation will succeed if and only if the predicates specified by
\code{checks} hold on the pre-existing object.  If any of the predicates are not
true for the existing object, then the operation will have no effect and fail
with \code{CMPFAIL}.

All checks are atomic with the write.  HyperDex guarantees that no other
operation will come between validating the checks, and writing the new version
of the object..



\paragraph{Definition:}
\begin{gocode}
func (client *Client) CondMapAtomicMul(spacename string, key Value, predicates []Predicate, mapattributes MapAttributes) (err *Error)
\end{gocode}

\paragraph{Parameters:}
\begin{itemize}[noitemsep]
\item \code{spacename}\\
The name of the space as a C-string.

\item \code{key}\\
The key for the operation where \code{key} is a Javascript value.

\item \code{predicates}\\
A set of predicates to check against.  \code{checks} is a map from the
attributes' names to the predicates to check.

\item \code{mapattributes}\\
The set of attributes to modify and their respective values.  \code{mapattrs} is
a map from the attributes' names to inner maps which contain the key-value pairs
to be modified

\end{itemize}

\paragraph{Returns:}
This function returns via the provided callback.  In the normal case, the first
argument will indicate success or failure of the operation with one of the
following values:

\begin{itemize}[noitemsep]
\item \code{true} if the operation succeeded
\item \code{false} if any provided predicates failed
\item \code{null} if the operation requires an existing value and none exist
\end{itemize}

If the operation encounters any error, the error argument will be provided and
will specify the error, in which case the first argument is undefined.


%%%%%%%%%%%%%%%%%%%% GroupMapAtomicMul %%%%%%%%%%%%%%%%%%%%
\pagebreak
\subsubsection{\code{GroupMapAtomicMul}}
\label{api:Go:GroupMapAtomicMul}
\index{GroupMapAtomicMul!Go API}
Multiply the value of each key-value pair by the specified number for each
object in \code{space} that matches \code{checks}.

This operation will only affect objects that match the provided \code{checks}.
Objects that do not match \code{checks} will be unaffected by the group call.
Each object that matches \code{checks} will be atomically updated with the check
on the object.  HyperDex guarantees that no object will be altered if the
\code{checks} do not pass at the time of the write.  Objects that are updated
concurrently with the group call may or may not be updated; however, regardless
of any other concurrent operations, the preceding guarantee will always hold.



\paragraph{Definition:}
\begin{gocode}
func (client *Client) GroupMapAtomicMul(spacename string, predicates []Predicate, mapattributes MapAttributes) (count uint64, err *Error)
\end{gocode}

\paragraph{Parameters:}
\begin{itemize}[noitemsep]
\item \code{spacename}\\
The name of the space as a C-string.

\item \code{predicates}\\
A set of predicates to check against.  \code{checks} is a map from the
attributes' names to the predicates to check.

\item \code{mapattributes}\\
The set of attributes to modify and their respective values.  \code{mapattrs} is
a map from the attributes' names to inner maps which contain the key-value pairs
to be modified

\end{itemize}

\paragraph{Returns:}
This function returns via the provided callback.  In the normal case, the first
argument will be the total number of objects that match the predicate.

If the operation encounters any error, the error argument will be provided and
will specify the error, in which case the first argument is undefined.


%%%%%%%%%%%%%%%%%%%% MapAtomicDiv %%%%%%%%%%%%%%%%%%%%
\pagebreak
\subsubsection{\code{MapAtomicDiv}}
\label{api:Go:MapAtomicDiv}
\index{MapAtomicDiv!Go API}
Divide the value of each key-value pair by the specified number for each map.
This operation requires a pre-existing object in order to complete successfully.
If no object exists, the operation will fail with \code{NOTFOUND}.



\paragraph{Definition:}
\begin{gocode}
func (client *Client) MapAtomicDiv(spacename string, key Value, mapattributes MapAttributes) (err *Error)
\end{gocode}

\paragraph{Parameters:}
\begin{itemize}[noitemsep]
\item \code{spacename}\\
The name of the space as a C-string.

\item \code{key}\\
The key for the operation where \code{key} is a Javascript value.

\item \code{mapattributes}\\
The set of attributes to modify and their respective values.  \code{mapattrs} is
a map from the attributes' names to inner maps which contain the key-value pairs
to be modified

\end{itemize}

\paragraph{Returns:}
This function returns via the provided callback.  In the normal case, the first
argument will indicate success or failure of the operation with one of the
following values:

\begin{itemize}[noitemsep]
\item \code{true} if the operation succeeded
\item \code{false} if any provided predicates failed
\item \code{null} if the operation requires an existing value and none exist
\end{itemize}

If the operation encounters any error, the error argument will be provided and
will specify the error, in which case the first argument is undefined.


%%%%%%%%%%%%%%%%%%%% CondMapAtomicDiv %%%%%%%%%%%%%%%%%%%%
\pagebreak
\subsubsection{\code{CondMapAtomicDiv}}
\label{api:Go:CondMapAtomicDiv}
\index{CondMapAtomicDiv!Go API}
Divide the value of each key-value pair by the specified number for each map if
and only if the \code{checks} hold on the object.
This operation requires a pre-existing object in order to complete successfully.
If no object exists, the operation will fail with \code{NOTFOUND}.


This operation will succeed if and only if the predicates specified by
\code{checks} hold on the pre-existing object.  If any of the predicates are not
true for the existing object, then the operation will have no effect and fail
with \code{CMPFAIL}.

All checks are atomic with the write.  HyperDex guarantees that no other
operation will come between validating the checks, and writing the new version
of the object..



\paragraph{Definition:}
\begin{gocode}
func (client *Client) CondMapAtomicDiv(spacename string, key Value, predicates []Predicate, mapattributes MapAttributes) (err *Error)
\end{gocode}

\paragraph{Parameters:}
\begin{itemize}[noitemsep]
\item \code{spacename}\\
The name of the space as a C-string.

\item \code{key}\\
The key for the operation where \code{key} is a Javascript value.

\item \code{predicates}\\
A set of predicates to check against.  \code{checks} is a map from the
attributes' names to the predicates to check.

\item \code{mapattributes}\\
The set of attributes to modify and their respective values.  \code{mapattrs} is
a map from the attributes' names to inner maps which contain the key-value pairs
to be modified

\end{itemize}

\paragraph{Returns:}
This function returns via the provided callback.  In the normal case, the first
argument will indicate success or failure of the operation with one of the
following values:

\begin{itemize}[noitemsep]
\item \code{true} if the operation succeeded
\item \code{false} if any provided predicates failed
\item \code{null} if the operation requires an existing value and none exist
\end{itemize}

If the operation encounters any error, the error argument will be provided and
will specify the error, in which case the first argument is undefined.


%%%%%%%%%%%%%%%%%%%% GroupMapAtomicDiv %%%%%%%%%%%%%%%%%%%%
\pagebreak
\subsubsection{\code{GroupMapAtomicDiv}}
\label{api:Go:GroupMapAtomicDiv}
\index{GroupMapAtomicDiv!Go API}
Divide the value of each key-value pair by the specified number for each object
in \code{space} that matches \code{checks}.

This operation will only affect objects that match the provided \code{checks}.
Objects that do not match \code{checks} will be unaffected by the group call.
Each object that matches \code{checks} will be atomically updated with the check
on the object.  HyperDex guarantees that no object will be altered if the
\code{checks} do not pass at the time of the write.  Objects that are updated
concurrently with the group call may or may not be updated; however, regardless
of any other concurrent operations, the preceding guarantee will always hold.



\paragraph{Definition:}
\begin{gocode}
func (client *Client) GroupMapAtomicDiv(spacename string, predicates []Predicate, mapattributes MapAttributes) (count uint64, err *Error)
\end{gocode}

\paragraph{Parameters:}
\begin{itemize}[noitemsep]
\item \code{spacename}\\
The name of the space as a C-string.

\item \code{predicates}\\
A set of predicates to check against.  \code{checks} is a map from the
attributes' names to the predicates to check.

\item \code{mapattributes}\\
The set of attributes to modify and their respective values.  \code{mapattrs} is
a map from the attributes' names to inner maps which contain the key-value pairs
to be modified

\end{itemize}

\paragraph{Returns:}
This function returns via the provided callback.  In the normal case, the first
argument will be the total number of objects that match the predicate.

If the operation encounters any error, the error argument will be provided and
will specify the error, in which case the first argument is undefined.


%%%%%%%%%%%%%%%%%%%% MapAtomicMod %%%%%%%%%%%%%%%%%%%%
\pagebreak
\subsubsection{\code{MapAtomicMod}}
\label{api:Go:MapAtomicMod}
\index{MapAtomicMod!Go API}
Store the value of the key-value pair modulo the specified number for each map.
This operation requires a pre-existing object in order to complete successfully.
If no object exists, the operation will fail with \code{NOTFOUND}.



\paragraph{Definition:}
\begin{gocode}
func (client *Client) MapAtomicMod(spacename string, key Value, mapattributes MapAttributes) (err *Error)
\end{gocode}

\paragraph{Parameters:}
\begin{itemize}[noitemsep]
\item \code{spacename}\\
The name of the space as a C-string.

\item \code{key}\\
The key for the operation where \code{key} is a Javascript value.

\item \code{mapattributes}\\
The set of attributes to modify and their respective values.  \code{mapattrs} is
a map from the attributes' names to inner maps which contain the key-value pairs
to be modified

\end{itemize}

\paragraph{Returns:}
This function returns via the provided callback.  In the normal case, the first
argument will indicate success or failure of the operation with one of the
following values:

\begin{itemize}[noitemsep]
\item \code{true} if the operation succeeded
\item \code{false} if any provided predicates failed
\item \code{null} if the operation requires an existing value and none exist
\end{itemize}

If the operation encounters any error, the error argument will be provided and
will specify the error, in which case the first argument is undefined.


%%%%%%%%%%%%%%%%%%%% CondMapAtomicMod %%%%%%%%%%%%%%%%%%%%
\pagebreak
\subsubsection{\code{CondMapAtomicMod}}
\label{api:Go:CondMapAtomicMod}
\index{CondMapAtomicMod!Go API}
Store the value of the key-value pair modulo the specified number for each map
attribute if and only if the \code{checks} hold on the object.
This operation requires a pre-existing object in order to complete successfully.
If no object exists, the operation will fail with \code{NOTFOUND}.


This operation will succeed if and only if the predicates specified by
\code{checks} hold on the pre-existing object.  If any of the predicates are not
true for the existing object, then the operation will have no effect and fail
with \code{CMPFAIL}.

All checks are atomic with the write.  HyperDex guarantees that no other
operation will come between validating the checks, and writing the new version
of the object..



\paragraph{Definition:}
\begin{gocode}
func (client *Client) CondMapAtomicMod(spacename string, key Value, predicates []Predicate, mapattributes MapAttributes) (err *Error)
\end{gocode}

\paragraph{Parameters:}
\begin{itemize}[noitemsep]
\item \code{spacename}\\
The name of the space as a C-string.

\item \code{key}\\
The key for the operation where \code{key} is a Javascript value.

\item \code{predicates}\\
A set of predicates to check against.  \code{checks} is a map from the
attributes' names to the predicates to check.

\item \code{mapattributes}\\
The set of attributes to modify and their respective values.  \code{mapattrs} is
a map from the attributes' names to inner maps which contain the key-value pairs
to be modified

\end{itemize}

\paragraph{Returns:}
This function returns via the provided callback.  In the normal case, the first
argument will indicate success or failure of the operation with one of the
following values:

\begin{itemize}[noitemsep]
\item \code{true} if the operation succeeded
\item \code{false} if any provided predicates failed
\item \code{null} if the operation requires an existing value and none exist
\end{itemize}

If the operation encounters any error, the error argument will be provided and
will specify the error, in which case the first argument is undefined.


%%%%%%%%%%%%%%%%%%%% GroupMapAtomicMod %%%%%%%%%%%%%%%%%%%%
\pagebreak
\subsubsection{\code{GroupMapAtomicMod}}
\label{api:Go:GroupMapAtomicMod}
\index{GroupMapAtomicMod!Go API}
Store the value of the key-value pair modulo the specified number for each
object in \code{space} that matches \code{checks}.

This operation will only affect objects that match the provided \code{checks}.
Objects that do not match \code{checks} will be unaffected by the group call.
Each object that matches \code{checks} will be atomically updated with the check
on the object.  HyperDex guarantees that no object will be altered if the
\code{checks} do not pass at the time of the write.  Objects that are updated
concurrently with the group call may or may not be updated; however, regardless
of any other concurrent operations, the preceding guarantee will always hold.



\paragraph{Definition:}
\begin{gocode}
func (client *Client) GroupMapAtomicMod(spacename string, predicates []Predicate, mapattributes MapAttributes) (count uint64, err *Error)
\end{gocode}

\paragraph{Parameters:}
\begin{itemize}[noitemsep]
\item \code{spacename}\\
The name of the space as a C-string.

\item \code{predicates}\\
A set of predicates to check against.  \code{checks} is a map from the
attributes' names to the predicates to check.

\item \code{mapattributes}\\
The set of attributes to modify and their respective values.  \code{mapattrs} is
a map from the attributes' names to inner maps which contain the key-value pairs
to be modified

\end{itemize}

\paragraph{Returns:}
This function returns via the provided callback.  In the normal case, the first
argument will be the total number of objects that match the predicate.

If the operation encounters any error, the error argument will be provided and
will specify the error, in which case the first argument is undefined.


%%%%%%%%%%%%%%%%%%%% MapAtomicAnd %%%%%%%%%%%%%%%%%%%%
\pagebreak
\subsubsection{\code{MapAtomicAnd}}
\label{api:Go:MapAtomicAnd}
\index{MapAtomicAnd!Go API}
Store the bitwise AND of the value of the key-value pair and the specified
number for each map.
This operation requires a pre-existing object in order to complete successfully.
If no object exists, the operation will fail with \code{NOTFOUND}.



\paragraph{Definition:}
\begin{gocode}
func (client *Client) MapAtomicAnd(spacename string, key Value, mapattributes MapAttributes) (err *Error)
\end{gocode}

\paragraph{Parameters:}
\begin{itemize}[noitemsep]
\item \code{spacename}\\
The name of the space as a C-string.

\item \code{key}\\
The key for the operation where \code{key} is a Javascript value.

\item \code{mapattributes}\\
The set of attributes to modify and their respective values.  \code{mapattrs} is
a map from the attributes' names to inner maps which contain the key-value pairs
to be modified

\end{itemize}

\paragraph{Returns:}
This function returns via the provided callback.  In the normal case, the first
argument will indicate success or failure of the operation with one of the
following values:

\begin{itemize}[noitemsep]
\item \code{true} if the operation succeeded
\item \code{false} if any provided predicates failed
\item \code{null} if the operation requires an existing value and none exist
\end{itemize}

If the operation encounters any error, the error argument will be provided and
will specify the error, in which case the first argument is undefined.


%%%%%%%%%%%%%%%%%%%% CondMapAtomicAnd %%%%%%%%%%%%%%%%%%%%
\pagebreak
\subsubsection{\code{CondMapAtomicAnd}}
\label{api:Go:CondMapAtomicAnd}
\index{CondMapAtomicAnd!Go API}
Store the bitwise AND of the value of the key-value pair and the specified
number for each map attribute if and only if the \code{checks} hold on the
object.
This operation requires a pre-existing object in order to complete successfully.
If no object exists, the operation will fail with \code{NOTFOUND}.


This operation will succeed if and only if the predicates specified by
\code{checks} hold on the pre-existing object.  If any of the predicates are not
true for the existing object, then the operation will have no effect and fail
with \code{CMPFAIL}.

All checks are atomic with the write.  HyperDex guarantees that no other
operation will come between validating the checks, and writing the new version
of the object..



\paragraph{Definition:}
\begin{gocode}
func (client *Client) CondMapAtomicAnd(spacename string, key Value, predicates []Predicate, mapattributes MapAttributes) (err *Error)
\end{gocode}

\paragraph{Parameters:}
\begin{itemize}[noitemsep]
\item \code{spacename}\\
The name of the space as a C-string.

\item \code{key}\\
The key for the operation where \code{key} is a Javascript value.

\item \code{predicates}\\
A set of predicates to check against.  \code{checks} is a map from the
attributes' names to the predicates to check.

\item \code{mapattributes}\\
The set of attributes to modify and their respective values.  \code{mapattrs} is
a map from the attributes' names to inner maps which contain the key-value pairs
to be modified

\end{itemize}

\paragraph{Returns:}
This function returns via the provided callback.  In the normal case, the first
argument will indicate success or failure of the operation with one of the
following values:

\begin{itemize}[noitemsep]
\item \code{true} if the operation succeeded
\item \code{false} if any provided predicates failed
\item \code{null} if the operation requires an existing value and none exist
\end{itemize}

If the operation encounters any error, the error argument will be provided and
will specify the error, in which case the first argument is undefined.


%%%%%%%%%%%%%%%%%%%% GroupMapAtomicAnd %%%%%%%%%%%%%%%%%%%%
\pagebreak
\subsubsection{\code{GroupMapAtomicAnd}}
\label{api:Go:GroupMapAtomicAnd}
\index{GroupMapAtomicAnd!Go API}
Store the bitwise AND of the value of the key-value pair and the specified
number for each map attribute for each object in \code{space} that matches
\code{checks}.

This operation will only affect objects that match the provided \code{checks}.
Objects that do not match \code{checks} will be unaffected by the group call.
Each object that matches \code{checks} will be atomically updated with the check
on the object.  HyperDex guarantees that no object will be altered if the
\code{checks} do not pass at the time of the write.  Objects that are updated
concurrently with the group call may or may not be updated; however, regardless
of any other concurrent operations, the preceding guarantee will always hold.



\paragraph{Definition:}
\begin{gocode}
func (client *Client) GroupMapAtomicAnd(spacename string, predicates []Predicate, mapattributes MapAttributes) (count uint64, err *Error)
\end{gocode}

\paragraph{Parameters:}
\begin{itemize}[noitemsep]
\item \code{spacename}\\
The name of the space as a C-string.

\item \code{predicates}\\
A set of predicates to check against.  \code{checks} is a map from the
attributes' names to the predicates to check.

\item \code{mapattributes}\\
The set of attributes to modify and their respective values.  \code{mapattrs} is
a map from the attributes' names to inner maps which contain the key-value pairs
to be modified

\end{itemize}

\paragraph{Returns:}
This function returns via the provided callback.  In the normal case, the first
argument will be the total number of objects that match the predicate.

If the operation encounters any error, the error argument will be provided and
will specify the error, in which case the first argument is undefined.


%%%%%%%%%%%%%%%%%%%% MapAtomicOr %%%%%%%%%%%%%%%%%%%%
\pagebreak
\subsubsection{\code{MapAtomicOr}}
\label{api:Go:MapAtomicOr}
\index{MapAtomicOr!Go API}
Store the bitwise OR of the value of the key-value pair and the specified number
for each map.
This operation requires a pre-existing object in order to complete successfully.
If no object exists, the operation will fail with \code{NOTFOUND}.



\paragraph{Definition:}
\begin{gocode}
func (client *Client) MapAtomicOr(spacename string, key Value, mapattributes MapAttributes) (err *Error)
\end{gocode}

\paragraph{Parameters:}
\begin{itemize}[noitemsep]
\item \code{spacename}\\
The name of the space as a C-string.

\item \code{key}\\
The key for the operation where \code{key} is a Javascript value.

\item \code{mapattributes}\\
The set of attributes to modify and their respective values.  \code{mapattrs} is
a map from the attributes' names to inner maps which contain the key-value pairs
to be modified

\end{itemize}

\paragraph{Returns:}
This function returns via the provided callback.  In the normal case, the first
argument will indicate success or failure of the operation with one of the
following values:

\begin{itemize}[noitemsep]
\item \code{true} if the operation succeeded
\item \code{false} if any provided predicates failed
\item \code{null} if the operation requires an existing value and none exist
\end{itemize}

If the operation encounters any error, the error argument will be provided and
will specify the error, in which case the first argument is undefined.


%%%%%%%%%%%%%%%%%%%% CondMapAtomicOr %%%%%%%%%%%%%%%%%%%%
\pagebreak
\subsubsection{\code{CondMapAtomicOr}}
\label{api:Go:CondMapAtomicOr}
\index{CondMapAtomicOr!Go API}
Store the bitwise OR of the value of the key-value pair and the specified number
for each map attribute if and only if the \code{checks} hold on the object.
This operation requires a pre-existing object in order to complete successfully.
If no object exists, the operation will fail with \code{NOTFOUND}.


This operation will succeed if and only if the predicates specified by
\code{checks} hold on the pre-existing object.  If any of the predicates are not
true for the existing object, then the operation will have no effect and fail
with \code{CMPFAIL}.

All checks are atomic with the write.  HyperDex guarantees that no other
operation will come between validating the checks, and writing the new version
of the object..



\paragraph{Definition:}
\begin{gocode}
func (client *Client) CondMapAtomicOr(spacename string, key Value, predicates []Predicate, mapattributes MapAttributes) (err *Error)
\end{gocode}

\paragraph{Parameters:}
\begin{itemize}[noitemsep]
\item \code{spacename}\\
The name of the space as a C-string.

\item \code{key}\\
The key for the operation where \code{key} is a Javascript value.

\item \code{predicates}\\
A set of predicates to check against.  \code{checks} is a map from the
attributes' names to the predicates to check.

\item \code{mapattributes}\\
The set of attributes to modify and their respective values.  \code{mapattrs} is
a map from the attributes' names to inner maps which contain the key-value pairs
to be modified

\end{itemize}

\paragraph{Returns:}
This function returns via the provided callback.  In the normal case, the first
argument will indicate success or failure of the operation with one of the
following values:

\begin{itemize}[noitemsep]
\item \code{true} if the operation succeeded
\item \code{false} if any provided predicates failed
\item \code{null} if the operation requires an existing value and none exist
\end{itemize}

If the operation encounters any error, the error argument will be provided and
will specify the error, in which case the first argument is undefined.


%%%%%%%%%%%%%%%%%%%% GroupMapAtomicOr %%%%%%%%%%%%%%%%%%%%
\pagebreak
\subsubsection{\code{GroupMapAtomicOr}}
\label{api:Go:GroupMapAtomicOr}
\index{GroupMapAtomicOr!Go API}
Store the bitwise OR of the value of the key-value pair and the specified number
for each map attribute for each object in \code{space} that matches
\code{checks}.

This operation will only affect objects that match the provided \code{checks}.
Objects that do not match \code{checks} will be unaffected by the group call.
Each object that matches \code{checks} will be atomically updated with the check
on the object.  HyperDex guarantees that no object will be altered if the
\code{checks} do not pass at the time of the write.  Objects that are updated
concurrently with the group call may or may not be updated; however, regardless
of any other concurrent operations, the preceding guarantee will always hold.



\paragraph{Definition:}
\begin{gocode}
func (client *Client) GroupMapAtomicOr(spacename string, predicates []Predicate, mapattributes MapAttributes) (count uint64, err *Error)
\end{gocode}

\paragraph{Parameters:}
\begin{itemize}[noitemsep]
\item \code{spacename}\\
The name of the space as a C-string.

\item \code{predicates}\\
A set of predicates to check against.  \code{checks} is a map from the
attributes' names to the predicates to check.

\item \code{mapattributes}\\
The set of attributes to modify and their respective values.  \code{mapattrs} is
a map from the attributes' names to inner maps which contain the key-value pairs
to be modified

\end{itemize}

\paragraph{Returns:}
This function returns via the provided callback.  In the normal case, the first
argument will be the total number of objects that match the predicate.

If the operation encounters any error, the error argument will be provided and
will specify the error, in which case the first argument is undefined.


%%%%%%%%%%%%%%%%%%%% MapAtomicXor %%%%%%%%%%%%%%%%%%%%
\pagebreak
\subsubsection{\code{MapAtomicXor}}
\label{api:Go:MapAtomicXor}
\index{MapAtomicXor!Go API}
Store the bitwise XOR of the value of the key-value pair and the specified
number for each map attribute.
This operation requires a pre-existing object in order to complete successfully.
If no object exists, the operation will fail with \code{NOTFOUND}.



\paragraph{Definition:}
\begin{gocode}
func (client *Client) MapAtomicXor(spacename string, key Value, mapattributes MapAttributes) (err *Error)
\end{gocode}

\paragraph{Parameters:}
\begin{itemize}[noitemsep]
\item \code{spacename}\\
The name of the space as a C-string.

\item \code{key}\\
The key for the operation where \code{key} is a Javascript value.

\item \code{mapattributes}\\
The set of attributes to modify and their respective values.  \code{mapattrs} is
a map from the attributes' names to inner maps which contain the key-value pairs
to be modified

\end{itemize}

\paragraph{Returns:}
This function returns via the provided callback.  In the normal case, the first
argument will indicate success or failure of the operation with one of the
following values:

\begin{itemize}[noitemsep]
\item \code{true} if the operation succeeded
\item \code{false} if any provided predicates failed
\item \code{null} if the operation requires an existing value and none exist
\end{itemize}

If the operation encounters any error, the error argument will be provided and
will specify the error, in which case the first argument is undefined.


%%%%%%%%%%%%%%%%%%%% CondMapAtomicXor %%%%%%%%%%%%%%%%%%%%
\pagebreak
\subsubsection{\code{CondMapAtomicXor}}
\label{api:Go:CondMapAtomicXor}
\index{CondMapAtomicXor!Go API}
Store the bitwise XOR of the value of the key-value pair and the specified
number for each map attribute if and only if the \code{checks} hold on the
object.
This operation requires a pre-existing object in order to complete successfully.
If no object exists, the operation will fail with \code{NOTFOUND}.


This operation will succeed if and only if the predicates specified by
\code{checks} hold on the pre-existing object.  If any of the predicates are not
true for the existing object, then the operation will have no effect and fail
with \code{CMPFAIL}.

All checks are atomic with the write.  HyperDex guarantees that no other
operation will come between validating the checks, and writing the new version
of the object..



\paragraph{Definition:}
\begin{gocode}
func (client *Client) CondMapAtomicXor(spacename string, key Value, predicates []Predicate, mapattributes MapAttributes) (err *Error)
\end{gocode}

\paragraph{Parameters:}
\begin{itemize}[noitemsep]
\item \code{spacename}\\
The name of the space as a C-string.

\item \code{key}\\
The key for the operation where \code{key} is a Javascript value.

\item \code{predicates}\\
A set of predicates to check against.  \code{checks} is a map from the
attributes' names to the predicates to check.

\item \code{mapattributes}\\
The set of attributes to modify and their respective values.  \code{mapattrs} is
a map from the attributes' names to inner maps which contain the key-value pairs
to be modified

\end{itemize}

\paragraph{Returns:}
This function returns via the provided callback.  In the normal case, the first
argument will indicate success or failure of the operation with one of the
following values:

\begin{itemize}[noitemsep]
\item \code{true} if the operation succeeded
\item \code{false} if any provided predicates failed
\item \code{null} if the operation requires an existing value and none exist
\end{itemize}

If the operation encounters any error, the error argument will be provided and
will specify the error, in which case the first argument is undefined.


%%%%%%%%%%%%%%%%%%%% GroupMapAtomicXor %%%%%%%%%%%%%%%%%%%%
\pagebreak
\subsubsection{\code{GroupMapAtomicXor}}
\label{api:Go:GroupMapAtomicXor}
\index{GroupMapAtomicXor!Go API}
Store the bitwise XOR of the value of the key-value pair and the specified
number for each map attribute for each object in \code{space} that matches
\code{checks}.

This operation will only affect objects that match the provided \code{checks}.
Objects that do not match \code{checks} will be unaffected by the group call.
Each object that matches \code{checks} will be atomically updated with the check
on the object.  HyperDex guarantees that no object will be altered if the
\code{checks} do not pass at the time of the write.  Objects that are updated
concurrently with the group call may or may not be updated; however, regardless
of any other concurrent operations, the preceding guarantee will always hold.



\paragraph{Definition:}
\begin{gocode}
func (client *Client) GroupMapAtomicXor(spacename string, predicates []Predicate, mapattributes MapAttributes) (count uint64, err *Error)
\end{gocode}

\paragraph{Parameters:}
\begin{itemize}[noitemsep]
\item \code{spacename}\\
The name of the space as a C-string.

\item \code{predicates}\\
A set of predicates to check against.  \code{checks} is a map from the
attributes' names to the predicates to check.

\item \code{mapattributes}\\
The set of attributes to modify and their respective values.  \code{mapattrs} is
a map from the attributes' names to inner maps which contain the key-value pairs
to be modified

\end{itemize}

\paragraph{Returns:}
This function returns via the provided callback.  In the normal case, the first
argument will be the total number of objects that match the predicate.

If the operation encounters any error, the error argument will be provided and
will specify the error, in which case the first argument is undefined.


%%%%%%%%%%%%%%%%%%%% MapStringPrepend %%%%%%%%%%%%%%%%%%%%
\pagebreak
\subsubsection{\code{MapStringPrepend}}
\label{api:Go:MapStringPrepend}
\index{MapStringPrepend!Go API}
Prepend the specified string to the value of the key-value pair for each map.
This operation requires a pre-existing object in order to complete successfully.
If no object exists, the operation will fail with \code{NOTFOUND}.



\paragraph{Definition:}
\begin{gocode}
func (client *Client) MapStringPrepend(spacename string, key Value, mapattributes MapAttributes) (err *Error)
\end{gocode}

\paragraph{Parameters:}
\begin{itemize}[noitemsep]
\item \code{spacename}\\
The name of the space as a C-string.

\item \code{key}\\
The key for the operation where \code{key} is a Javascript value.

\item \code{mapattributes}\\
The set of attributes to modify and their respective values.  \code{mapattrs} is
a map from the attributes' names to inner maps which contain the key-value pairs
to be modified

\end{itemize}

\paragraph{Returns:}
This function returns via the provided callback.  In the normal case, the first
argument will indicate success or failure of the operation with one of the
following values:

\begin{itemize}[noitemsep]
\item \code{true} if the operation succeeded
\item \code{false} if any provided predicates failed
\item \code{null} if the operation requires an existing value and none exist
\end{itemize}

If the operation encounters any error, the error argument will be provided and
will specify the error, in which case the first argument is undefined.


%%%%%%%%%%%%%%%%%%%% CondMapStringPrepend %%%%%%%%%%%%%%%%%%%%
\pagebreak
\subsubsection{\code{CondMapStringPrepend}}
\label{api:Go:CondMapStringPrepend}
\index{CondMapStringPrepend!Go API}
Prepend the specified string to the value of the key-value pair for each map
attribute if and only if the \code{checks} hold on the object.
This operation requires a pre-existing object in order to complete successfully.
If no object exists, the operation will fail with \code{NOTFOUND}.


This operation will succeed if and only if the predicates specified by
\code{checks} hold on the pre-existing object.  If any of the predicates are not
true for the existing object, then the operation will have no effect and fail
with \code{CMPFAIL}.

All checks are atomic with the write.  HyperDex guarantees that no other
operation will come between validating the checks, and writing the new version
of the object..



\paragraph{Definition:}
\begin{gocode}
func (client *Client) CondMapStringPrepend(spacename string, key Value, predicates []Predicate, mapattributes MapAttributes) (err *Error)
\end{gocode}

\paragraph{Parameters:}
\begin{itemize}[noitemsep]
\item \code{spacename}\\
The name of the space as a C-string.

\item \code{key}\\
The key for the operation where \code{key} is a Javascript value.

\item \code{predicates}\\
A set of predicates to check against.  \code{checks} is a map from the
attributes' names to the predicates to check.

\item \code{mapattributes}\\
The set of attributes to modify and their respective values.  \code{mapattrs} is
a map from the attributes' names to inner maps which contain the key-value pairs
to be modified

\end{itemize}

\paragraph{Returns:}
This function returns via the provided callback.  In the normal case, the first
argument will indicate success or failure of the operation with one of the
following values:

\begin{itemize}[noitemsep]
\item \code{true} if the operation succeeded
\item \code{false} if any provided predicates failed
\item \code{null} if the operation requires an existing value and none exist
\end{itemize}

If the operation encounters any error, the error argument will be provided and
will specify the error, in which case the first argument is undefined.


%%%%%%%%%%%%%%%%%%%% GroupMapStringPrepend %%%%%%%%%%%%%%%%%%%%
\pagebreak
\subsubsection{\code{GroupMapStringPrepend}}
\label{api:Go:GroupMapStringPrepend}
\index{GroupMapStringPrepend!Go API}
Prepend the specified string to the value of the key-value pair for each map
attribute for each object in \code{space} that matches \code{checks}.

This operation will only affect objects that match the provided \code{checks}.
Objects that do not match \code{checks} will be unaffected by the group call.
Each object that matches \code{checks} will be atomically updated with the check
on the object.  HyperDex guarantees that no object will be altered if the
\code{checks} do not pass at the time of the write.  Objects that are updated
concurrently with the group call may or may not be updated; however, regardless
of any other concurrent operations, the preceding guarantee will always hold.



\paragraph{Definition:}
\begin{gocode}
func (client *Client) GroupMapStringPrepend(spacename string, predicates []Predicate, mapattributes MapAttributes) (count uint64, err *Error)
\end{gocode}

\paragraph{Parameters:}
\begin{itemize}[noitemsep]
\item \code{spacename}\\
The name of the space as a C-string.

\item \code{predicates}\\
A set of predicates to check against.  \code{checks} is a map from the
attributes' names to the predicates to check.

\item \code{mapattributes}\\
The set of attributes to modify and their respective values.  \code{mapattrs} is
a map from the attributes' names to inner maps which contain the key-value pairs
to be modified

\end{itemize}

\paragraph{Returns:}
This function returns via the provided callback.  In the normal case, the first
argument will be the total number of objects that match the predicate.

If the operation encounters any error, the error argument will be provided and
will specify the error, in which case the first argument is undefined.


%%%%%%%%%%%%%%%%%%%% MapStringAppend %%%%%%%%%%%%%%%%%%%%
\pagebreak
\subsubsection{\code{MapStringAppend}}
\label{api:Go:MapStringAppend}
\index{MapStringAppend!Go API}
Append the specified string to the value of the key-value pair for each map
attribute.
This operation requires a pre-existing object in order to complete successfully.
If no object exists, the operation will fail with \code{NOTFOUND}.



\paragraph{Definition:}
\begin{gocode}
func (client *Client) MapStringAppend(spacename string, key Value, mapattributes MapAttributes) (err *Error)
\end{gocode}

\paragraph{Parameters:}
\begin{itemize}[noitemsep]
\item \code{spacename}\\
The name of the space as a C-string.

\item \code{key}\\
The key for the operation where \code{key} is a Javascript value.

\item \code{mapattributes}\\
The set of attributes to modify and their respective values.  \code{mapattrs} is
a map from the attributes' names to inner maps which contain the key-value pairs
to be modified

\end{itemize}

\paragraph{Returns:}
This function returns via the provided callback.  In the normal case, the first
argument will indicate success or failure of the operation with one of the
following values:

\begin{itemize}[noitemsep]
\item \code{true} if the operation succeeded
\item \code{false} if any provided predicates failed
\item \code{null} if the operation requires an existing value and none exist
\end{itemize}

If the operation encounters any error, the error argument will be provided and
will specify the error, in which case the first argument is undefined.


%%%%%%%%%%%%%%%%%%%% CondMapStringAppend %%%%%%%%%%%%%%%%%%%%
\pagebreak
\subsubsection{\code{CondMapStringAppend}}
\label{api:Go:CondMapStringAppend}
\index{CondMapStringAppend!Go API}
Append the specified string to the value of the key-value pair for each map
attribute if and only if the \code{checks} hold on the object.
This operation requires a pre-existing object in order to complete successfully.
If no object exists, the operation will fail with \code{NOTFOUND}.


This operation will succeed if and only if the predicates specified by
\code{checks} hold on the pre-existing object.  If any of the predicates are not
true for the existing object, then the operation will have no effect and fail
with \code{CMPFAIL}.

All checks are atomic with the write.  HyperDex guarantees that no other
operation will come between validating the checks, and writing the new version
of the object..



\paragraph{Definition:}
\begin{gocode}
func (client *Client) CondMapStringAppend(spacename string, key Value, predicates []Predicate, mapattributes MapAttributes) (err *Error)
\end{gocode}

\paragraph{Parameters:}
\begin{itemize}[noitemsep]
\item \code{spacename}\\
The name of the space as a C-string.

\item \code{key}\\
The key for the operation where \code{key} is a Javascript value.

\item \code{predicates}\\
A set of predicates to check against.  \code{checks} is a map from the
attributes' names to the predicates to check.

\item \code{mapattributes}\\
The set of attributes to modify and their respective values.  \code{mapattrs} is
a map from the attributes' names to inner maps which contain the key-value pairs
to be modified

\end{itemize}

\paragraph{Returns:}
This function returns via the provided callback.  In the normal case, the first
argument will indicate success or failure of the operation with one of the
following values:

\begin{itemize}[noitemsep]
\item \code{true} if the operation succeeded
\item \code{false} if any provided predicates failed
\item \code{null} if the operation requires an existing value and none exist
\end{itemize}

If the operation encounters any error, the error argument will be provided and
will specify the error, in which case the first argument is undefined.


%%%%%%%%%%%%%%%%%%%% GroupMapStringAppend %%%%%%%%%%%%%%%%%%%%
\pagebreak
\subsubsection{\code{GroupMapStringAppend}}
\label{api:Go:GroupMapStringAppend}
\index{GroupMapStringAppend!Go API}
Append the specified string to the value of the key-value pair for each map
attribute for each object in \code{space} that matches \code{checks}

This operation will only affect objects that match the provided \code{checks}.
Objects that do not match \code{checks} will be unaffected by the group call.
Each object that matches \code{checks} will be atomically updated with the check
on the object.  HyperDex guarantees that no object will be altered if the
\code{checks} do not pass at the time of the write.  Objects that are updated
concurrently with the group call may or may not be updated; however, regardless
of any other concurrent operations, the preceding guarantee will always hold.



\paragraph{Definition:}
\begin{gocode}
func (client *Client) GroupMapStringAppend(spacename string, predicates []Predicate, mapattributes MapAttributes) (count uint64, err *Error)
\end{gocode}

\paragraph{Parameters:}
\begin{itemize}[noitemsep]
\item \code{spacename}\\
The name of the space as a C-string.

\item \code{predicates}\\
A set of predicates to check against.  \code{checks} is a map from the
attributes' names to the predicates to check.

\item \code{mapattributes}\\
The set of attributes to modify and their respective values.  \code{mapattrs} is
a map from the attributes' names to inner maps which contain the key-value pairs
to be modified

\end{itemize}

\paragraph{Returns:}
This function returns via the provided callback.  In the normal case, the first
argument will be the total number of objects that match the predicate.

If the operation encounters any error, the error argument will be provided and
will specify the error, in which case the first argument is undefined.


%%%%%%%%%%%%%%%%%%%% MapAtomicMin %%%%%%%%%%%%%%%%%%%%
\pagebreak
\subsubsection{\code{MapAtomicMin}}
\label{api:Go:MapAtomicMin}
\index{MapAtomicMin!Go API}
Take the minium of the specified value and existing value for each key-value
pair.
This operation requires a pre-existing object in order to complete successfully.
If no object exists, the operation will fail with \code{NOTFOUND}.



\paragraph{Definition:}
\begin{gocode}
func (client *Client) MapAtomicMin(spacename string, key Value, mapattributes MapAttributes) (err *Error)
\end{gocode}

\paragraph{Parameters:}
\begin{itemize}[noitemsep]
\item \code{spacename}\\
The name of the space as a C-string.

\item \code{key}\\
The key for the operation where \code{key} is a Javascript value.

\item \code{mapattributes}\\
The set of attributes to modify and their respective values.  \code{mapattrs} is
a map from the attributes' names to inner maps which contain the key-value pairs
to be modified

\end{itemize}

\paragraph{Returns:}
This function returns via the provided callback.  In the normal case, the first
argument will indicate success or failure of the operation with one of the
following values:

\begin{itemize}[noitemsep]
\item \code{true} if the operation succeeded
\item \code{false} if any provided predicates failed
\item \code{null} if the operation requires an existing value and none exist
\end{itemize}

If the operation encounters any error, the error argument will be provided and
will specify the error, in which case the first argument is undefined.


%%%%%%%%%%%%%%%%%%%% CondMapAtomicMin %%%%%%%%%%%%%%%%%%%%
\pagebreak
\subsubsection{\code{CondMapAtomicMin}}
\label{api:Go:CondMapAtomicMin}
\index{CondMapAtomicMin!Go API}
Take the minium of the specified value and existing value for each key-value
pair if and only if the \code{checks} hold on the object.
This operation requires a pre-existing object in order to complete successfully.
If no object exists, the operation will fail with \code{NOTFOUND}.


This operation will succeed if and only if the predicates specified by
\code{checks} hold on the pre-existing object.  If any of the predicates are not
true for the existing object, then the operation will have no effect and fail
with \code{CMPFAIL}.

All checks are atomic with the write.  HyperDex guarantees that no other
operation will come between validating the checks, and writing the new version
of the object..



\paragraph{Definition:}
\begin{gocode}
func (client *Client) CondMapAtomicMin(spacename string, key Value, predicates []Predicate, mapattributes MapAttributes) (err *Error)
\end{gocode}

\paragraph{Parameters:}
\begin{itemize}[noitemsep]
\item \code{spacename}\\
The name of the space as a C-string.

\item \code{key}\\
The key for the operation where \code{key} is a Javascript value.

\item \code{predicates}\\
A set of predicates to check against.  \code{checks} is a map from the
attributes' names to the predicates to check.

\item \code{mapattributes}\\
The set of attributes to modify and their respective values.  \code{mapattrs} is
a map from the attributes' names to inner maps which contain the key-value pairs
to be modified

\end{itemize}

\paragraph{Returns:}
This function returns via the provided callback.  In the normal case, the first
argument will indicate success or failure of the operation with one of the
following values:

\begin{itemize}[noitemsep]
\item \code{true} if the operation succeeded
\item \code{false} if any provided predicates failed
\item \code{null} if the operation requires an existing value and none exist
\end{itemize}

If the operation encounters any error, the error argument will be provided and
will specify the error, in which case the first argument is undefined.


%%%%%%%%%%%%%%%%%%%% GroupMapAtomicMin %%%%%%%%%%%%%%%%%%%%
\pagebreak
\subsubsection{\code{GroupMapAtomicMin}}
\label{api:Go:GroupMapAtomicMin}
\index{GroupMapAtomicMin!Go API}
Take the minium of the specified value and existing value for each key-value
pair for each object in \code{space} that matches \code{checks}.

This operation will only affect objects that match the provided \code{checks}.
Objects that do not match \code{checks} will be unaffected by the group call.
Each object that matches \code{checks} will be atomically updated with the check
on the object.  HyperDex guarantees that no object will be altered if the
\code{checks} do not pass at the time of the write.  Objects that are updated
concurrently with the group call may or may not be updated; however, regardless
of any other concurrent operations, the preceding guarantee will always hold.



\paragraph{Definition:}
\begin{gocode}
func (client *Client) GroupMapAtomicMin(spacename string, predicates []Predicate, mapattributes MapAttributes) (count uint64, err *Error)
\end{gocode}

\paragraph{Parameters:}
\begin{itemize}[noitemsep]
\item \code{spacename}\\
The name of the space as a C-string.

\item \code{predicates}\\
A set of predicates to check against.  \code{checks} is a map from the
attributes' names to the predicates to check.

\item \code{mapattributes}\\
The set of attributes to modify and their respective values.  \code{mapattrs} is
a map from the attributes' names to inner maps which contain the key-value pairs
to be modified

\end{itemize}

\paragraph{Returns:}
This function returns via the provided callback.  In the normal case, the first
argument will be the total number of objects that match the predicate.

If the operation encounters any error, the error argument will be provided and
will specify the error, in which case the first argument is undefined.


%%%%%%%%%%%%%%%%%%%% MapAtomicMax %%%%%%%%%%%%%%%%%%%%
\pagebreak
\subsubsection{\code{MapAtomicMax}}
\label{api:Go:MapAtomicMax}
\index{MapAtomicMax!Go API}
Take the maximum of the specified value and existing value for each key-value
pair.
This operation requires a pre-existing object in order to complete successfully.
If no object exists, the operation will fail with \code{NOTFOUND}.



\paragraph{Definition:}
\begin{gocode}
func (client *Client) MapAtomicMax(spacename string, key Value, mapattributes MapAttributes) (err *Error)
\end{gocode}

\paragraph{Parameters:}
\begin{itemize}[noitemsep]
\item \code{spacename}\\
The name of the space as a C-string.

\item \code{key}\\
The key for the operation where \code{key} is a Javascript value.

\item \code{mapattributes}\\
The set of attributes to modify and their respective values.  \code{mapattrs} is
a map from the attributes' names to inner maps which contain the key-value pairs
to be modified

\end{itemize}

\paragraph{Returns:}
This function returns via the provided callback.  In the normal case, the first
argument will indicate success or failure of the operation with one of the
following values:

\begin{itemize}[noitemsep]
\item \code{true} if the operation succeeded
\item \code{false} if any provided predicates failed
\item \code{null} if the operation requires an existing value and none exist
\end{itemize}

If the operation encounters any error, the error argument will be provided and
will specify the error, in which case the first argument is undefined.


%%%%%%%%%%%%%%%%%%%% CondMapAtomicMax %%%%%%%%%%%%%%%%%%%%
\pagebreak
\subsubsection{\code{CondMapAtomicMax}}
\label{api:Go:CondMapAtomicMax}
\index{CondMapAtomicMax!Go API}
Take the maximum of the specified value and existing value for each key-value
pair if and only if the \code{checks} hold on the object.
This operation requires a pre-existing object in order to complete successfully.
If no object exists, the operation will fail with \code{NOTFOUND}.


This operation will succeed if and only if the predicates specified by
\code{checks} hold on the pre-existing object.  If any of the predicates are not
true for the existing object, then the operation will have no effect and fail
with \code{CMPFAIL}.

All checks are atomic with the write.  HyperDex guarantees that no other
operation will come between validating the checks, and writing the new version
of the object..



\paragraph{Definition:}
\begin{gocode}
func (client *Client) CondMapAtomicMax(spacename string, key Value, predicates []Predicate, mapattributes MapAttributes) (err *Error)
\end{gocode}

\paragraph{Parameters:}
\begin{itemize}[noitemsep]
\item \code{spacename}\\
The name of the space as a C-string.

\item \code{key}\\
The key for the operation where \code{key} is a Javascript value.

\item \code{predicates}\\
A set of predicates to check against.  \code{checks} is a map from the
attributes' names to the predicates to check.

\item \code{mapattributes}\\
The set of attributes to modify and their respective values.  \code{mapattrs} is
a map from the attributes' names to inner maps which contain the key-value pairs
to be modified

\end{itemize}

\paragraph{Returns:}
This function returns via the provided callback.  In the normal case, the first
argument will indicate success or failure of the operation with one of the
following values:

\begin{itemize}[noitemsep]
\item \code{true} if the operation succeeded
\item \code{false} if any provided predicates failed
\item \code{null} if the operation requires an existing value and none exist
\end{itemize}

If the operation encounters any error, the error argument will be provided and
will specify the error, in which case the first argument is undefined.


%%%%%%%%%%%%%%%%%%%% GroupMapAtomicMax %%%%%%%%%%%%%%%%%%%%
\pagebreak
\subsubsection{\code{GroupMapAtomicMax}}
\label{api:Go:GroupMapAtomicMax}
\index{GroupMapAtomicMax!Go API}
Take the maximum of the specified value and existing value for each key-value
pair for each object in \code{space} that matches \code{checks}.

This operation will only affect objects that match the provided \code{checks}.
Objects that do not match \code{checks} will be unaffected by the group call.
Each object that matches \code{checks} will be atomically updated with the check
on the object.  HyperDex guarantees that no object will be altered if the
\code{checks} do not pass at the time of the write.  Objects that are updated
concurrently with the group call may or may not be updated; however, regardless
of any other concurrent operations, the preceding guarantee will always hold.



\paragraph{Definition:}
\begin{gocode}
func (client *Client) GroupMapAtomicMax(spacename string, predicates []Predicate, mapattributes MapAttributes) (count uint64, err *Error)
\end{gocode}

\paragraph{Parameters:}
\begin{itemize}[noitemsep]
\item \code{spacename}\\
The name of the space as a C-string.

\item \code{predicates}\\
A set of predicates to check against.  \code{checks} is a map from the
attributes' names to the predicates to check.

\item \code{mapattributes}\\
The set of attributes to modify and their respective values.  \code{mapattrs} is
a map from the attributes' names to inner maps which contain the key-value pairs
to be modified

\end{itemize}

\paragraph{Returns:}
This function returns via the provided callback.  In the normal case, the first
argument will be the total number of objects that match the predicate.

If the operation encounters any error, the error argument will be provided and
will specify the error, in which case the first argument is undefined.


%%%%%%%%%%%%%%%%%%%% Search %%%%%%%%%%%%%%%%%%%%
\pagebreak
\subsubsection{\code{Search}}
\label{api:Go:Search}
\index{Search!Go API}
Return all objects that match the specified \code{checks}.
This operation behaves as an iterator and may return multiple objects from the
single call.



\paragraph{Definition:}
\begin{gocode}
func (client *Client) Search(spacename string, predicates []Predicate) (attrs chan Attributes, errs chan Error)
\end{gocode}

\paragraph{Parameters:}
\begin{itemize}[noitemsep]
\item \code{spacename}\\
The name of the space as a C-string.

\item \code{predicates}\\
A set of predicates to check against.  \code{checks} is a map from the
attributes' names to the predicates to check.

\end{itemize}

\paragraph{Returns:}
An iterator that returns objects one at a time.  Excpetions will be returned via
the same path to enable applications to retrieve partial results in the face of
errors.


%%%%%%%%%%%%%%%%%%%% SortedSearch %%%%%%%%%%%%%%%%%%%%
\pagebreak
\subsubsection{\code{SortedSearch}}
\label{api:Go:SortedSearch}
\index{SortedSearch!Go API}
Return all objects that match the specified \code{checks}, sorted according to
\code{attr}.
This operation behaves as an iterator and may return multiple objects from the
single call.



\paragraph{Definition:}
\begin{gocode}
func (client *Client) SortedSearch(spacename string, predicates []Predicate, sortby string, limit uint32, maxmin string) (attrs chan Attributes, errs chan Error)
\end{gocode}

\paragraph{Parameters:}
\begin{itemize}[noitemsep]
\item \code{spacename}\\
The name of the space as a C-string.

\item \code{predicates}\\
A set of predicates to check against.  \code{checks} is a map from the
attributes' names to the predicates to check.

\item \code{sortby}\\
The attribute to sort by.

\item \code{limit}\\
The number of results to return.

\item \code{maxmin}\\
Maximize (\code{'max'}) or minimize (\code{'min'}).

\end{itemize}

\paragraph{Returns:}
An iterator that returns objects one at a time.  Excpetions will be returned via
the same path to enable applications to retrieve partial results in the face of
errors.


%%%%%%%%%%%%%%%%%%%% Count %%%%%%%%%%%%%%%%%%%%
\pagebreak
\subsubsection{\code{Count}}
\label{api:Go:Count}
\index{Count!Go API}
Count the number of objects that match the specified \code{checks}.  This will
return the number of objects counted by the search.  If an error occurs during
the count, the count may reflect a partial count.  The real count will be higher
than the returned value.

A count is the server-side equivalent of performing a search, and counting the
number of objects that exist.  For that reason, its efficiency closely follows
that of search, except it will not transfer any objects over the network.


\paragraph{Definition:}
\begin{gocode}
func (client *Client) Count(spacename string, predicates []Predicate) (count uint64, err *Error)
\end{gocode}

\paragraph{Parameters:}
\begin{itemize}[noitemsep]
\item \code{spacename}\\
The name of the space as a C-string.

\item \code{predicates}\\
A set of predicates to check against.  \code{checks} is a map from the
attributes' names to the predicates to check.

\end{itemize}

\paragraph{Returns:}
This function returns via the provided callback.  In the normal case, the first
argument will be the total number of objects that match the predicate.

If the operation encounters any error, the error argument will be provided and
will specify the error, in which case the first argument is undefined.


\pagebreak

\subsection{Working with Signals}
\label{sec:api:java:signals}

The HyperDex client library is signal-safe.  Should a signal interrupt the
client during a blocking operation, it will raise a
\code{HyperDexClientException} with status \code{HYPERDEX\_CLIENT\_INTERRUPTED}.

\subsection{Working with Threads}
\label{sec:api:Java:threads}

The Java package is fully reentrant.  Instances of
\code{HyperDex::Client::Client} and their associated state may be accessed from
multiple threads, provided that the application employs its own synchronization
that provides mutual exclusion.

Put simply, a multi-threaded application should protect each \code{Client}
instance with a mutex or lock to ensure correct operation.
